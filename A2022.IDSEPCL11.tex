\documentclass{beamer}
%
% Choose how your presentation looks.
% For more themes, color themes and font themes, see:
% http://deic.uab.es/~iblanes/beamer_gallery/index_by_theme.html
%
\mode<presentation>
{
  \usetheme{Madrid}      % or try Darmstadt, Madrid, Warsaw, ...
  \usecolortheme{seahorse} % or try albatross, beaver, crane, ...
  \usefonttheme{serif}  % or try serif, structurebold, ...
  \setbeamertemplate{navigation symbols}{}
  \setbeamertemplate{caption}[numbered]
  \usepackage{amsmath}
  \usepackage{tcolorbox}
  \usepackage[export]{adjustbox}
  \tcbuselibrary{most}
  \usepackage{amsmath}  % conflicts with \oiint
  \usepackage{arydshln}
  \usepackage{tikz}
  \usetikzlibrary{plotmarks}
  \usepackage{pgfplots}
  \newcommand{\Ivec}[1]{\mbox{\boldmath $#1$}}
    \usepackage[normalem]{ulem}
    % \usepackage{tipa}
    \newcommand{\git}{\texttt{\textbf{git}}\xspace}
    \usepackage{listings}
} 


\definecolor{myblue}{RGB}{65,105,225} 
\definecolor{myorange}{RGB}{250,190,0}

\setbeamercolor{structure}{fg=white,bg=myorange}
\setbeamercolor*{palette primary}{fg=myblue,bg=myorange}
\setbeamercolor*{palette secondary}{fg=white,bg=myblue}
\setbeamercolor*{palette tertiary}{bg=myblue,fg=white}
\setbeamercolor*{palette quaternary}{fg=white,bg=myorange!50}

\setbeamercolor{frametitle}{fg=black!90!myblue}

\setbeamercolor{section in head/foot}{fg=white,bg=myblue}
\setbeamercolor{author in head/foot}{fg=black,bg=myorange}
\setbeamercolor{title in head/foot}{fg=white,bg=myblue}

\setbeamertemplate{navigation symbols}{}

\setbeamertemplate{itemize/enumerate body begin}{\large}
\setbeamertemplate{itemize/enumerate subbody begin}{\large}


\defbeamertemplate*{headline}{mytheme}
{%
  \begin{beamercolorbox}[ht=2.25ex,dp=3.75ex]{section in head/foot}
    \insertnavigation{\paperwidth}
  \end{beamercolorbox}%
}%

\defbeamertemplate*{footline}{mytheme}
{
  \leavevmode%
  \hbox{%
  \begin{beamercolorbox}[wd=.5\paperwidth,ht=2.25ex,dp=1ex,right]{author in head/foot}%
    \usebeamerfont{author in head/foot}\insertshortauthor\hspace*{2em}
  \end{beamercolorbox}%
  \begin{beamercolorbox}[wd=.5\paperwidth,ht=2.25ex,dp=1ex,left]{title in head/foot}%
    \usebeamerfont{title in head/foot}\hspace*{2em}\insertshortsubtitle\hspace*{2em}
    \insertframenumber{} / \inserttotalframenumber
  \end{beamercolorbox}}%
  \vskip0pt%
}

\usepackage[english]{babel}
\usepackage[utf8x]{inputenc}
\usepackage{xcolor}
\usepackage{listings}
\usepackage{pgf}  
\usepackage{textpos}
\usepackage{tabulary}
\usepackage{scrextend}
\usepackage{hyperref}
\usepackage{setspace}
\usepackage{rotating}
\lstset
{
    language=[LaTeX]TeX,
    breaklines=true,
    basicstyle=\tt\scriptsize,
    %commentstyle=\color{green}
    keywordstyle=\color{blue},
    %stringstyle=\color{black}
    identifierstyle=\color{magenta},
}
\newcommand{\bftt}[1]{\textbf{\texttt{#1}}}
%\newcommand{\comment}[1]{{\color[HTML]{008080}\textit{\textbf{\texttt{#1}}}}}
\newcommand{\cmd}[1]{{\color[HTML]{008000}\bftt{#1}}}
\newcommand{\bs}{\char`\\}
\newcommand{\cmdbs}[1]{\cmd{\bs#1}}
\newcommand{\lcb}{\char '173}
\newcommand{\rcb}{\char '175}
\newcommand{\cmdbegin}[1]{\cmdbs{begin\lcb}\bftt{#1}\cmd{\rcb}}
\newcommand{\cmdend}[1]{\cmdbs{end\lcb}\bftt{#1}\cmd{\rcb}}

\newcommand{\wllogo}{\textbf{Overleaf}}

\usepackage{hyperref}
\hypersetup{
    colorlinks=true,
    linkcolor=blue,
    filecolor=magenta,      
    urlcolor=cyan,
    pdftitle={Overleaf Example},
    pdfpagemode=FullScreen,
    }

\urlstyle{same}

% this is where the example source files are loaded from
% do not include a trailing slash
\newcommand{\fileuri}{https://raw.githubusercontent.com/GiancarloSucci/UniBo.IDSEPC.A2022/main/A2022.IDSEPCLaTeX/}

\newcommand{\perche}{perch\'{e}}
\newcommand{\affinche}{affinch\'{e}}
\newcommand{\se}{s\'{e}}
\newcommand{\a}{\`{a}}
\newcommand{\est}{\`{e}}

\usepackage{stackengine}
\def\Ruble{\stackengine{.67ex}{%
  \stackengine{.48ex}{\textsf{P}}{\rule{.8ex}{.12ex}\kern.6ex}{O}{r}{F}{F}{L}%
  }{\rule{.8ex}{.12ex}\kern.6ex}{O}{r}{F}{F}{L}\kern-.1ex}



%----------------------------------------------------------------------------------------
%	TITLE PAGE
%----------------------------------------------------------------------------------------
\title[L02]{Introduzione alla data science e al pensiero computazionale\\
Lezione 11: Regressione Logistica} % The short title appears at the bottom of every slide, the full title is only on the title page

\author[{\tiny Giancarlo Succi }]{Giancarlo Succi\\\\ Dipartimento di Informatica -- Scienza e Ingegneria\\Universit\`{a} di Bologna\\
\bftt{g.succi@unibo.it}
} % Your name
\institute[unibo] % Your institution as it will appear on the bottom of every slide, may be shorthand to save space

\date{} % Date, can be changed to a custom date

\setbeamertemplate{navigation symbols}{}
\AtBeginSection[]
{
        \begin{frame}<beamer>{Outline}
                \tableofcontents[currentsection]
        \end{frame}
}

\begin{document}
\begin{frame}
\titlepage % Print the title page as the first slide

\end{frame}

\begin{frame}
{\centerline{Content}}

\begin{itemize}
\item Covariance
\item Correlation (aka Pearson product-moment correlation coefficient)
\item Relationship between Pearson correlation and linear regression
\end{itemize}
\end{frame}


\begin{frame}
{\centerline{Outline}}
\begin{itemize}
\item Likelihood function, definition
\item Maximum likelihood
\item Log likelihood
\item Logistic regression
\end{itemize}

\vspace*{2cm}
Some slides are take from: \url{https://www.cs.ox.ac.uk/people/nando.defreitas/}
\end{frame}


%----------------------------------------------------------------------------------------
%------------------------------------------------
\begin{frame}
{\centerline{Likelihood function }}
Let $X_1,X_2,...,X_n$ denote a random sample from p.d.f.
$$X_i \sim f_{\theta}(x),$$
where $\theta$ represents one ore more unknown parameters of the distribution.

The joint p.d.f. of $X_1, X_2, ..., X_n$ is $f_{\theta}(x_1),f_{\theta}(x_2),...,f_{\theta}(x_n)$.

If we consider this joint p.d.f. as a function of $\theta$ it is called \textit{likelihood function} of a random sample:
$$L_{x_1,x_2,...,x_n}(\theta)=f_{\theta}(x_1),f_{\theta}(x_2),...,f_{\theta}(x_n).$$
\end{frame}

\begin{frame}
{\centerline{Maximum likelihood}}
Let's consider an estimator of $\theta$:
$$\hat{\theta} = u(X_1,X_2,...,X_n).$$

If for every possible $\theta$ $L_{x_1,x_2,...,x_n}(\hat{\theta})$ is at least as great as $L_{x_1,x_2,...,x_n}(\theta)$ then $\hat{\theta}$ is called \textit{maximum likelihood estimator}.

Finally:
$$\hat{\theta} = \argmax\limits_{\theta}(L_{x_1,x_2,...,x_n}(\theta))$$

\end{frame}

\begin{frame}
{\centerline{Maximum loglikelihood}}
Note that, since the likelihood function $L_{x_1,x_2,...,x_n}(\theta)$ and its logarithm $ln(L_{x_1,x_2,...,x_n}(\theta))$, are maximized for the same value $\theta$, either likelihood or its logarithm can be used to find maximum likelihood estimator:
$$\hat{\theta} = \argmax\limits_{\theta}(ln(L_{x_1,x_2,...,x_n}(\theta)))$$
\end{frame}

\begin{frame}
{\centerline{The concept of regression}}

Regressions can be of multiple types, so far we have analysed the so called OLS regression:
\begin{itemize}
\item quadratic cost function of the kind $\sum_i (\hat{y}_i - y_i)^2$
\item linear model of the kind $\hat{y} =  \mathbf{A} \mathbf{x} + \mathbf{\eta}$
\end{itemize}

\vspace*{1cm}
What if:
\begin{itemize}
\item we use a different objective function, or
\item we use a different model
\end{itemize}
\centerline{\textcolor{red}{\Huge \bf ?}}
\vspace*{1cm}
\textit{Remember that model is called ``the \textcolor{red}{mean} function'' and its inverse ``the \textcolor{red}{link} function.''}
\end{frame}

\begin{frame}
{\centerline{Posing a different problem}}
Let's suppose to have:
\begin{itemize}
\item three iid random variables $y_i$ with $ i \in [1 \ldots 3]$
\item with the same partially unknown pdf, that is
\item ($\forall i$) $y_i \thicksim N(\theta,1)$
\item $\theta$ to be determined.
\end{itemize}

We want to determine the value of $\theta$ that maximizes the probability of obtaining $y_1$ and $y_2$ and $y_3$.\\

\vspace*{0.5cm}
In other terms our objective function is the probability of occurrence of $y_1$ and $y_2$ and $y_3$.

\vspace*{0.5cm}
We are looking for a maximum likelihood estimator!

\end{frame}

\begin{frame}
{\centerline{Computing the highest probability}}
Our objective function is therefore:
$$P(y_1, y_2, y_3 | \theta) = P(y_1 | \theta) \times P(y_2 | \theta) \times P(y_3 | \theta)$$

We can rewrite this problem as:

$$ \max_{\theta}(\prod_{i=1}^3 P(y_i | \theta)) $$
%\textcolor{blue}{careful! since $\theta$ is not a random variable, $P(y_i | \theta))$ =?}
\vspace*{0.2cm}

Note that since $\theta$ is a \textit{crisp} value:
\centerline{
$ y_i \thicksim N(\theta,1) = $ a shift of $\theta$ of $N(0,1) $}


\end{frame}

\begin{frame}
{\centerline{Using concrete numbers - 1}}
Let us assume that:
\begin{itemize}
\item $y_1 = 1$
\item $y_2 = 0.5$
\item $y_3 = 1.5$
\end{itemize}
Remember that $N(\theta,\sigma) = \dfrac 1 {\sigma \sqrt{2 \pi} } \, e^{-\dfrac { \left({y - \theta}\right)^2} {2 \sigma^2} }$\\
\vspace{0.5cm}

Therefore, we want to maximize:\\
$ \prod_{i=1}^3 P(y_i | \theta) = \prod_{i=1}^3 \dfrac 1 {\sqrt{2 \pi} } \, e^{-\dfrac { \left({y_i - \theta}\right)^2} {2} } = $\\
\vspace{0.5cm}

$= \dfrac 1 {\sqrt{2 \pi} } \, e^{-\dfrac { \left({1 - \theta}\right)^2} {2} } \times  \dfrac 1 {\sqrt{2 \pi} } \, e^{-\dfrac { \left({0.5 - \theta}\right)^2} {2} } \times  \dfrac 1 {\sqrt{2 \pi} } \, e^{-\dfrac { \left({1.5 - \theta}\right)^2} {2} }$

\end{frame}

\begin{frame}
{\centerline{Using concrete numbers - 2}}

This is like maximizing:
\textcolor{blue}{
$$e^{-\dfrac { \left({1 - \theta}\right)^2} {2} } \times e^{-\dfrac { \left({0.5 - \theta}\right)^2} {2} } \times e^{-\dfrac { \left({1.5 - \theta}\right)^2} {2} } = $$
}\textcolor{cyan}{
$$ = e^{-\dfrac { \left({1 - \theta}\right)^2} {2} -\dfrac { \left({0.5 - \theta}\right)^2} {2} -\dfrac { \left({1.5 - \theta}\right)^2} {2}} = $$
}\textcolor{red}{
$$ = e^{-\dfrac { ({1 - \theta})^2 + ({0.5 - \theta})^2 + ({1.5 - \theta})^2 } {2} } = e^{-\dfrac { 3.5 - 6 \theta + 3 \theta ^2 } {2} } $$
}
This is like minimizing $ g(\theta) = 3.5 - 6 \theta + 3 \theta ^2 $.
$$\frac{d g(\theta)}{d \theta} = \frac{d 3.5 - 6 \theta + 3 \theta ^2}{d \theta} = -6 + 6 \theta$$
Which becomes 0 for $\theta = 1$

\end{frame}

\begin{frame}
{\centerline{What we have discovered }}
Our solution is therefore $\theta = 1$ and the desired pdf is $N(1,1)$. But ... 
$$mean(1, 0,5, 1.5) = 1$$

\vspace*{3.5cm}
We can try to generalize it...

\end{frame}

\begin{frame}
{\centerline{Generalizing ... }}

Let's suppose to have:
\begin{itemize}
\item $n$ iid random variables $y_i$ with $ i \in [1 \ldots n]$
\item with the same partially unknown pdf, that is
\item ($\forall i$) $y_i \thicksim N(\theta,\sigma)$
\item $\theta$ and $\sigma$ to be determined.
\end{itemize}

We want to determine the value of $\theta$ that maximizes the probability of obtaining ($\forall i$) $y_i$.\\

\vspace*{0.5cm}
In other terms our objective is to maximize the probability of occurrence of all $y_i$, that is a maximum likelihood estimation.

\vspace*{0.5cm}
Typically, we would perform a least square estimation, and we know that optimal least square estimator is the Gaussian centered in the average of the points, with their standard deviation.
\end{frame}




\begin{frame}
{\centerline{Maximum likelihood estimator (again) }}
Let' look for a maximum likelihood estimator!
$$ \max_{\sigma, \theta}(\prod_{i=1}^n P(y_i | \sigma, \theta)) = \max_{\sigma, \theta}(\prod_{i=1}^n P(y_i | \sigma, \theta)) =  \max_{\sigma, \theta}(\prod_{i=1}^n \dfrac 1 {\sigma \sqrt{2 \pi} } \, e^{-\dfrac { ({y_i - \theta})^2} {2 \sigma^2} })=$$

$$ = \max_{\sigma, \theta} \big((\dfrac 1 {\sigma \sqrt{2 \pi} } )^{n} \prod_{i=1}^n  e^{-\dfrac { ({y_i - \theta})^2} {2 \sigma^2} }\big))=
\max_{\sigma, \theta} \big((\dfrac 1 {\sigma \sqrt{2 \pi} } )^{n} e^{- \sum_{i=1}^n  \dfrac { ({y_i - \theta})^2} {2 \sigma^2} }) = $$

$$  =
\max_{\sigma, \theta} \big((\dfrac 1 {\sigma \sqrt{2 \pi} } )^{n} e^{-   \dfrac { 1 } {2 \sigma^2} \sum_{i=1}^n  ({y_i - \theta})^2} ) $$

At this point we can take the log of the expression, knowing that the log function is differentiable and monotonically increasing on all $\mathbb{R}$.

\end{frame}


\begin{frame}
{\centerline{Computing the ml estimator - 1}}

$$log\big((\dfrac 1 {\sigma \sqrt{2 \pi} } )^{n} e^{-   \dfrac { 1 } {2 \sigma^2} \sum_{i=1}^n  ({y_i - \theta})^2} ) =$$

$$= n \times log (\dfrac 1 {\sigma \sqrt{2 \pi} }) + log(e^{ - \dfrac { 1 } {2 \sigma^2} \sum_{i=1}^n  ({y_i - \theta})^2}) =$$

$$= n \times log (\dfrac 1 {\sigma \sqrt{2 \pi} }) - \dfrac { 1 } {2 \sigma^2} \times \sum_{i=1}^n  ({y_i - \theta})^2 $$

Taking the partial derivative over $\theta$ we obtain: 

$$\frac{\partial \Big(n \times log (\dfrac 1 {\sigma \sqrt{2 \pi} }) - \dfrac { 1 } {2 \sigma^2} \times \sum_{i=1}^n  ({y_i - \theta})^2 \Big)}{\partial \theta} = $$


\end{frame}

\begin{frame}
{\centerline{Computing the ml estimator - $\theta$}}

$$ = - \frac{\partial \Big( \dfrac { 1 } {2 \sigma^2} \times \sum_{i=1}^n  ({y_i - \theta})^2 \Big)}{\partial \theta} = - \dfrac { 1 } {\sigma^2} \times \Big( \sum_{i=1}^n  ({y_i - \theta}) \Big) $$
And equating it to 0:
$$ - \dfrac { 1 } {\sigma^2} \times \Big( \sum_{i=1}^n  ({y_i - \theta}) \Big) = 0 ~\textcolor{red}{\Rightarrow}~
 \sum_{i=1}^n y_i = n \times \theta 
~\textcolor{red}{\Rightarrow}~
 \theta = \frac {\sum_{i=1}^n y_i} n  $$
 
\textcolor{red}{\bf Oh!} $\theta$ is the average of the observed $y_i$!

\end{frame}

\begin{frame}
{\centerline{Computing the ml estimator - $\sigma$ - 1}}
\textcolor{cyan}{
$$\frac{\partial \Big(n \times log (\dfrac 1 {\sigma \sqrt{2 \pi} }) - \dfrac { 1 } {2 \sigma^2} \times \sum_{i=1}^n  ({y_i - \theta})^2 \Big)}{\partial \sigma} = $$
}\textcolor{blue}{
$$ = \frac{\partial\Big( n \times log (\dfrac 1 {\sigma \sqrt{2 \pi} })\Big) }{\partial \sigma} - \frac{\partial \Big( \dfrac { 1 } {2 \sigma^2} \times \sum_{i=1}^n  ({y_i - \theta})^2 \Big)}{\partial \sigma} = $$
}\textcolor{red}{
$$  = - \frac n {\sigma} + \frac 1 {\sigma^3} \times \sum_{i=1}^n  ({y_i - \theta})^2$$
}
And equating it to 0:

$$ 
- \frac n {\sigma} + \frac 1 {\sigma^3} \times \sum_{i=1}^n  ({y_i - \theta})^2 = 0 
~\textcolor{red}{\Rightarrow}~
 \Big(\sum_{i=1}^n  ({y_i - \theta})^2 \Big) \times  \frac 1 {\sigma^3} = \frac n {\sigma} 
$$

\end{frame}

\begin{frame}
{\centerline{Computing the ml estimator - $\sigma$ - 2}}

Assuming $\sigma \ne 0$: 
$$ 
~\textcolor{red}{\Rightarrow}~
 \Big(\sum_{i=1}^n  ({y_i - \theta})^2 \Big) = n \times \sigma^2 
 ~\textcolor{red}{\Rightarrow}~
$$
But we know $\theta = \overline{y_i}$, therefore:
$$
~\textcolor{red}{\Rightarrow}~
 \sigma^2 = {\frac 1 n} \times \Big(\sum_{i=1}^n  ({y_i - \overline{y_i}})^2 \Big) 
 $$
 
\textcolor{red}{\bf Oh!} $\sigma$ is the standard deviation of the observed $y_i$!

\end{frame}

\begin{frame}
{\centerline{What we have found}}

We have determined that the maximum likelihood estimator for a sequence of points assumed to be distributed normally is formed by a normal distribution with:
\begin{itemize}
\item average equal to the average of the sample,
\item standard deviation equal to the standard derivation of the sample.
\end{itemize}

This coincides with the best quadratic estimator!
\vspace{1cm}

We now move forward considering the maximum likelihood estimator for a regression line, meaning, what happens if now we want to model an interdependencies using as objective function the maximum likelihood.

\end{frame}

\begin{frame}
{\centerline{Ml linear regression - HPs}}

Let's suppose to have:
\begin{itemize}
\item $n \times m$ values $x_{i,j}$ with $ i \in [1 \ldots n]$, $ j \in [1 \ldots m]$ represented in short by a matrix $\boldsymbol{X}$ or a vector $\mathbf{x_i}$, $n>m$ \textit{(why?)}
\item $n$ iid random variables $y_i$ with $ i \in [1 \ldots n]$ represented in short by a vector $\boldsymbol{y}$
\item a linear relationships $\boldsymbol{\theta}$ between $\boldsymbol{X}$  and $\boldsymbol{y}$, \textit{that is, we use the usual \textcolor{red}{link / mean} functions}
\item each $y_i$ distributed normally with mean $\boldsymbol{x_i^T \theta}$ and standard deviation $\sigma$ (the same $\sigma$ for all $y_i$), that is 
\item ($\forall i$) $y_i \thicksim N(\boldsymbol{x_i^T \theta},\sigma)$
\item $\boldsymbol{\theta}$ and $\sigma$ to be determined.
\end{itemize}
\end{frame}

\begin{frame}
{\centerline{Ml linear regression - goals}}

We want to determine the value of $\boldsymbol{\theta}$ and $\sigma$ that maximizes the probability of obtaining ($\forall i$) $y_i$, that is:

$$ \max_{\theta, \sigma}( P(\mathbf{y} | \boldsymbol{X}, \boldsymbol{\theta}, \sigma)) = \max_{\boldsymbol{\theta}, \sigma}(\prod_{i=1}^n P(y_i | \boldsymbol{x_i}, \boldsymbol{\theta}, \sigma)) $$


\vspace*{0.5cm}
In other terms, our objective function is the conditional probability of occurrence of all $y_i$.


\end{frame}

\begin{frame}
{\centerline{Computing the optimal $\theta$ - 1}}
We can express for simplicity our equation in vectorial form:
$$\max_{\sigma, \boldsymbol{\theta}}\left ( \left (\dfrac 1 {\sigma \sqrt{2 \pi} } \right)^n \, e^{-\dfrac { (\boldsymbol{y} - \boldsymbol{X\theta})^T (\boldsymbol{y} - \boldsymbol{X\theta})} {2 \sigma^2} }\right)$$

As mentioned, this is equivalent to maximizing the log:

$$\max_{\sigma, \boldsymbol{\theta}} \left( log \left (\left (\dfrac 1 {\sigma \sqrt{2 \pi} } \right )^n \, e^{-\dfrac { (\boldsymbol{y} - \boldsymbol{X\theta})^T (\boldsymbol{y} - \boldsymbol{X\theta})} {2 \sigma^2} }\right) \right )$$

Which becomes:
$$\max_{\sigma, \boldsymbol{\theta}} \left( n \times log \left (\dfrac 1 {\sigma \sqrt{2 \pi} } \right ) + log \left( \, e^{-\dfrac { (\boldsymbol{y} - \boldsymbol{X\theta})^T (\boldsymbol{y} - \boldsymbol{X\theta})} {2 \sigma^2} }\right) \right )$$

\end{frame}

\begin{frame}
{\centerline{Computing the optimal $\theta$ - 2}}

$$\max_{\sigma, \boldsymbol{\theta}} \left( n \times log \left (\dfrac 1 {\sqrt{2 \pi} } \right ) + n \times log \left (\dfrac 1 {\sigma } \right ) -\dfrac { (\boldsymbol{y} - \boldsymbol{X\theta})^T (\boldsymbol{y} - \boldsymbol{X\theta})} {2 \sigma^2} \right )$$

And now we take the partial derivative over $\boldsymbol{\theta}$:

$$\frac{\partial \left( n \times log \left (\dfrac 1 {\sqrt{2 \pi} } \right ) + n \times log \left (\dfrac 1 {\sigma } \right ) -\dfrac { (\boldsymbol{y} - \boldsymbol{X\theta})^T (\boldsymbol{y} - \boldsymbol{X\theta})} {2 \sigma^2} \right )} {\partial \boldsymbol{\theta}} = $$

$$= - \frac 1 {2 \sigma^2} \frac{\partial \left( { (\boldsymbol{y} - \boldsymbol{X\theta})^T (\boldsymbol{y} - \boldsymbol{X\theta})} \right )} {\partial \boldsymbol{\theta}} =  - \frac 1 {\sigma^2}  { (\boldsymbol{y} - \boldsymbol{X\theta})}  $$
And equating it to 0 we obtain:
$$ - \frac 1 {\sigma^2}  { (\boldsymbol{y} - \boldsymbol{X\theta})} = 0
~~~\textcolor{red}{\Rightarrow}~~~
\boldsymbol{y} = \boldsymbol{X\theta}
$$



\end{frame}

\begin{frame}
{\centerline{Computing the optimal $\theta$ - 3}}

If $\boldsymbol{X}$ were square, then the solution would be:
$$ \boldsymbol{\theta} = \boldsymbol{X^{-1}y}$$

But, as we said, $n > m$, therefore the solution is given by:

$$ \boldsymbol{\theta} = \boldsymbol{{(X^{T}X})^{-1}X^{T}y}$$

\vspace{2cm}

What a surprise, isn't it?
\end{frame}

\begin{frame}
{\centerline{Computing the optimal $\sigma$ }}
Starting from: 
$$\max_{\sigma, \boldsymbol{\theta}} \left( n \times log \left (\dfrac 1 {\sqrt{2 \pi} } \right ) + n \times log \left (\dfrac 1 {\sigma } \right ) -\dfrac { (\boldsymbol{y} - \boldsymbol{X\theta})^T (\boldsymbol{y} - \boldsymbol{X\theta})} {2 \sigma^2} \right )$$

And now we take the partial derivative over $\sigma$:

$$\frac{\partial \left( n \times log \left (\dfrac 1 {\sqrt{2 \pi} } \right ) + n \times log \left (\dfrac 1 {\sigma } \right ) -\dfrac { (\boldsymbol{y} - \boldsymbol{X\theta})^T (\boldsymbol{y} - \boldsymbol{X\theta})} {2 \sigma^2} \right )} {\partial \sigma} = $$

$$= - \dfrac n {\sigma } +  \dfrac { (\boldsymbol{y} - \boldsymbol{X\theta})^T (\boldsymbol{y} - \boldsymbol{X\theta})} { \sigma^3 } $$
And equating it to 0, assuming as usual $\sigma \ne 0$ we obtain:
$$ \sigma^2 = \dfrac { (\boldsymbol{y} - \boldsymbol{X\theta})^T (\boldsymbol{y} - \boldsymbol{X\theta})} { n }
$$


\end{frame}

\begin{frame}
{\centerline{Maximum likelihood estimator - properties}}
\underline{Claim 1:} The maximum likelihood estimator of a Gaussian distribution over a set of points coincides with the OLS estimator.\\
\underline{Proof:} See above.\\
\underline{QED}

\vspace{2cm}
\underline{Claim 2:} The maximum likelihood linear regression coincides with the OLS linear regression.\\
\underline{Proof:} See above.\\
\underline{QED}


\end{frame}


\begin{frame}
{\centerline{Bernoulli and maximum likelihood}}

The pdf of a Bernulli distribution can be represented in terms of conditional probability as:
$$ P(x | \theta) = \theta^x (1-\theta)^{(1-x)} $$
where clearly $x$ can only be 0 or 1.
\newline

We can now introduce the concept of entropy, already hinted in class. Entropy represents the level of uncertainty of a variable.


\end{frame}

\begin{frame}
{\centerline{Entropy (and Bernulli and ml)}}

\underline{Definition (Entropy):} Given a random vectorial variable $\boldmath{x}$ of $n$ components and a parameter $\theta$, we define entropy of $\boldmath{x}$, $H(\boldmath{x})$ as:
$$H(\boldmath{x}) = \sum_{i=1}^{n} p(x_i|\theta) \times log (p(x_i|\theta)) $$

We notice that for a Bernulli distribution:
$$ H(\boldmath{x}) = (1-\theta) log (1-\theta) + \theta log(\theta)$$

Indeed, as $\theta$ tends to 0 or to 1 the uncertainty tends to 0, since the likely value of $\boldmath{x}$ tend to be 0 or 1 respectively.

\end{frame}

\begin{frame}
{\centerline{From B\&B plus ml to LR}}
We are now ready to move to study a radically different form or regression, the so-called logistic regression.
\newline

Our goal is to have a regression that not only represents a relationship between two variables, but is also possible to capture a prediction of probability.
\newline

However, the value of a probability is from 0 to 1, so we need a ``good'' function that can translate any value in such range.
\newline

We use often as such function the so-called ``sigmoid function.'' To introduce the sigmoid we start with the definition of a ``logistic function.''

\end{frame}

\begin{frame}
{\centerline{Logistic}}
\underline{Definition (Logistic function):} Given $L, x_0 \in \mathbb{R}$, $k \in \mathbb{R}^+$ a
logistic function $f(x)$ is defined as: 
$$ f(x) = \dfrac{L}{1+e^{-k(x-x_0)}}$$
\underline{Properties (of the logistic function:)}
\begin{itemize}
\item the domain is all $\mathbb{R}$
\item the range is $[0 \ldots L]$ if $L$ is positive and $[L \ldots 0]$ if $L$ is negative
\item $f(x)$ is continuous, monotonically increasing, and differentiable  over all its domain
\item $f(x)$ is symmetric over $x_0$
\item $k$ is the rate of growth of $f(x)$ and for $k \to +\infty$ $f(x)$ tends to become the step function in $x_0$
\end{itemize}

\end{frame}

\begin{frame}
{\centerline{Sigmoid}}
\underline{Definition (Sigmoid):} Given $k \in \mathbb{R}^+$, a
sigmoid function $sigm(x)$ is defined as a logistic function with $L = 1$ and $x_0 = 0$: 
$$ sigm(x) = \dfrac{1}{1+e^{-kx}}$$
\underline{Properties (of the sigmoid function):}
\begin{itemize}
\item the domain is all $\mathbb{R}$
\item the range is $[0 \ldots 1]$ 
\item $sigm(x)$ is continuous, monotonically increasing, and differentiable  over all its domain
\item $sigm(x)$ is symmetric over $0$
\item $k$ is the rate of growth of $sigm(x)$ and for $k \to +\infty$ $sigm(x)$ tends to become the step function
\end{itemize}

\end{frame}

\begin{frame}
{\centerline{Toward a logistic regression}}

Suppose that we want to determine if a given event is going to happen based on a series of $n$ predictors $x_1 \ldots x_n$. We can model the probability of occurrence of the event with a random variable $y$.
\newline

It is as if we have a sequence of flipping of coins each with different values of the possible variables that affect the result, for instance the intensity of the flipping, the temperature, the wind, etc.
\newline

Based on such set we want to  predict what will be the result of the next flipping, given a set of values assigned to the covariates.
\newline

Our question is what is:
\begin{center}
P(Head $|$ strong toss, strong wind, 60 degrees)\\

{\Huge \textcolor{red}{\bf ?}}
\end{center}



\end{frame}

\begin{frame}
{\centerline{Toward a logistic regression}}
Let's try to build a regression line.
\newline

As we mentioned, any time we compute a regression we need to determine:
\begin{itemize}
\item the function to use as a model, and in this case a linear function would not be suitable, since probabilities range from 0 to 1, for this reason we select a \textcolor{red}{sigmoid function};
\item the objective function, and in this case the least square would be inappropriate because it is not a proper metrics space, so we opt for maximizing the conditional probability, that is, we aim at a \textcolor{red}{maximum likelihood} estimation.
\end{itemize}



\end{frame}

\begin{frame}
{\centerline{{\centerline{Logistic regression - HPs}}}}
Let 
\begin{itemize}
\item $(y_i,\boldmath{x_i})$  be a collection of pairs with:
\begin{itemize}
\item $i \in [1 \ldots n]$
\item $y_i \in \{0,1\}$
\item $\boldmath{x_i} \in \mathbb{R}^{m}$
\item $n > m$
\end{itemize}
\item assume that the $y_i$ are iid random variables
\item consider as target \textcolor{red}{mean} function the sigmoid
\item consider as optimality criteria the maximum likelihood
\end{itemize}

\end{frame}

\begin{frame}
{\centerline{{\centerline{Logistic regression - goals}}}}

We want to determine the values of the parameters that maximize the probability of obtaining ($\forall i$) $y_i$, that is:

$$ \max_{\text{\bf \it Parameters}}( P(\mathbf{y} | \boldsymbol{X}, \text{\bf \it Parameters}) = \max_{\boldsymbol{\theta}}(\prod_{i=1}^n P(y_i | \boldsymbol{x_i}, \text{\bf \it Parameters})) $$


\vspace*{0.5cm}
In other terms, our objective function is the conditional probability of occurrence of all $y_i$.
\vspace*{0.5cm}

Given our link/mean: 
$$ \max_{\theta}( P(\mathbf{y} | \boldsymbol{X}, \boldsymbol{\theta})) = \max_{\boldsymbol{\theta}}(\prod_{i=1}^n P(y_i | sigm(\boldsymbol{x_i}^{T} \boldsymbol{\theta})) $$

\end{frame}

\begin{frame}
{\centerline{{\centerline{Logistic regression - structure}}}}

Since the pdf of a Bernulli distribution is:
$$ P(z | k) = k^z (1-k)^{(1-z)} $$
For us the probability k of each event is ``approximated'' by the sigmoid function (our mean function):

$$ \boldsymbol{k} = \dfrac{1}{1+e^{-\boldsymbol{x_i}^{T} \boldsymbol{\theta}}}$$

And this lead us to

$$ P(y_i | \boldsymbol{x_i}, \boldsymbol{\theta}) = \left ( \dfrac{1}{1+e^{-\boldsymbol{x_i}^{T} \boldsymbol{\theta}}} \right )^{y_i} \times \left (1 - \dfrac{1}{1+e^{-\boldsymbol{x_i}^{T} \boldsymbol{\theta}}} \right )^{1-y_i}  $$


\end{frame}

\begin{frame}
{\centerline{{\centerline{Logistic regression - the problem }}}}

Our problem has therefore the form of:
$$ \max_{\theta}( P(\mathbf{y} | \boldsymbol{X}, \boldsymbol{\theta})) =  \max_{\boldsymbol{\theta}} \prod_{i=1}^n \left ( \dfrac{1}{1+e^{-\boldsymbol{x_i}^{T} \boldsymbol{\theta}}} \right )^{y_i} \times \left ( 1-\dfrac{1}{1+e^{-\boldsymbol{x_i}^{T} \boldsymbol{\theta}}} \right )^{1-y_i} $$

\textit{It is like finding an n-dimensional hyperplane dividing the n-dimensional hyperspace in 2 parts, those leading to y being 0 and those leading to y being 1.}

\vspace*{3cm}

\end{frame}


\begin{frame}
{\centerline{{\centerline{Logistic regression - solution - 1}}}}
Since the log function is continuous, differentiable and monotonically increasing in all ${\mathbb{R}^+}$, our problem is equivalent to:
$$ \max_{\boldsymbol{\theta}} \left ( log \left ( \prod_{i=1}^n \left ( \dfrac{1}{1+e^{-\boldsymbol{x_i}^{T} \boldsymbol{\theta}}} \right )^{y_i} \times \left (1 - \dfrac{1}{1+e^{-\boldsymbol{x_i}^{T} \boldsymbol{\theta}}} \right )^{1-y_i} \right ) \right )$$

And, given the property of logs, this is like maximizing:
\textcolor{blue}{
$$ log \left ( \prod_{i=1}^n \left ( \dfrac{1}{1+e^{-\boldsymbol{x_i}^{T} \boldsymbol{\theta}}} \right )^{y_i}  \right ) + log \left (  \prod_{i=1}^n \left (1 - \dfrac{1}{1+e^{-\boldsymbol{x_i}^{T} \boldsymbol{\theta}}} \right )^{1-y_i} \right ) = $$
}\textcolor{red}{
$$ = \sum_{i=1}^n log \left ( \dfrac{1}{1+e^{-\boldsymbol{x_i}^{T} \boldsymbol{\theta}}} \right )^{y_i}  + \sum_{i=1}^n log \left (1 - \dfrac{1}{1+e^{-\boldsymbol{x_i}^{T} \boldsymbol{\theta}}} \right )^{1-y_i} = $$
}
\end{frame}
\begin{frame}
{\centerline{{\centerline{Logistic regression - solution - 2}}}}
\textcolor{gray}{
$$  = \sum_{i=1}^n y_i \times log \left ( \dfrac{1}{1+e^{-\boldsymbol{x_i}^{T} \boldsymbol{\theta}}} \right ) + \sum_{i=1}^n (1-y_i) \times log \left (1 - \dfrac{1}{1+e^{-\boldsymbol{x_i}^{T} \boldsymbol{\theta}}} \right ) = \cdots{} $$
}
A bit of logarithms...
\textcolor{blue}{
 $$log \left ( \dfrac{1}{1+e^{-\boldsymbol{x_i}^{T} \boldsymbol{\theta}}} \right ) = log (1) - log \left( 1+e^{-\boldsymbol{x_i}^{T} \boldsymbol{\theta}} \right ) = -log \left( 1+e^{-\boldsymbol{x_i}^{T} \boldsymbol{\theta}} \right )$$
 }
 
 \textcolor{cyan}{
 $$log \left ( 1 - \dfrac{1}{1+e^{-\boldsymbol{x_i}^{T} \boldsymbol{\theta}}} \right ) = log \left ( \dfrac{1 + e^{-\boldsymbol{x_i}^{T} \boldsymbol{\theta}} - 1}{1+e^{-\boldsymbol{x_i}^{T} \boldsymbol{\theta}}} \right ) = log \left ( \dfrac{e^{-\boldsymbol{x_i}^{T} \boldsymbol{\theta}}}{1+e^{-\boldsymbol{x_i}^{T} \boldsymbol{\theta}}} \right ) =$$
$$= log \left ( e^{-\boldsymbol{x_i}^{T}\boldsymbol{\theta}} \right ) - log \left( 1+e^{-\boldsymbol{x_i}^{T} \boldsymbol{\theta}} \right ) =  \boldsymbol{x_i}^{T}\boldsymbol{\theta} - log \left( 1+e^{-\boldsymbol{x_i}^{T} \boldsymbol{\theta}} \right ) = $$
 }

\end{frame}

\begin{frame}
{\centerline{{\centerline{Logistic regression - solution - 3}}}}

$$ = - \sum_{i=1}^n y_i \times log \left( 1+e^{-\boldsymbol{x_i}^{T} \boldsymbol{\theta}} \right ) + \sum_{i=1}^n (1-y_i) \times  \left ( \boldsymbol{x_i}^{T}\boldsymbol{\theta} - log \left( 1+e^{-\boldsymbol{x_i}^{T} \boldsymbol{\theta}} \right ) \right ) = $$
For simplicity let $w_i$ be $log \left( 1+e^{-\boldsymbol{x_i}^{T} \boldsymbol{\theta}} \right )$.
\textcolor{red}{
$$
= - \sum_{i=1}^n y_i \times w_i + \sum_{i=1}^n \boldsymbol{x_i}^{T}\boldsymbol{\theta} - \sum_{i=1}^n w_i - \sum_{i=1}^n y_i \times \boldsymbol{x_i}^{T}\boldsymbol{\theta} + \sum_{i=1}^n y_i \times w_i =
$$
}\textcolor{orange}{
$$
= \sum_{i=1}^n \boldsymbol{x_i}^{T}\boldsymbol{\theta} - \sum_{i=1}^n w_i - \sum_{i=1}^n y_i \times \boldsymbol{x_i}^{T}\boldsymbol{\theta}  =
$$
}\textcolor{blue}{
$$
= \sum_{i=1}^n (1 + y_i) \boldsymbol{x_i}^{T}\boldsymbol{\theta} - \sum_{i=1}^n w_i 
$$
}
\end{frame}

\begin{frame}
{\centerline{{\centerline{Logistic regression - comments }}}}

Let $f(\boldmath{\theta})$ be:
$$
\textcolor{red}{ \sum_{i=1}^n (1 + y_i) \boldsymbol{x_i}^{T}\boldsymbol{\theta} } - 
\textcolor{blue}{\sum_{i=1}^n log \left( 1+e^{-\boldsymbol{x_i}^{T} \boldsymbol{\theta}} \right ) }
$$
\underline{Claim:} $f(\boldmath{\theta})$ is convex.
\newline

\underline{Proof:} Omitted
\newline

\underline{Consequence:}
Optimization algorithms can easily find the maximum.


\end{frame}



\end{document}
