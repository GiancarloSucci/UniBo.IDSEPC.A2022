\documentclass{beamer}
\input{A2022.IDSEPCLaTeX/preambolo}

%
% Choose how your presentation looks.
% For more themes, color themes and font themes, see:
% http://deic.uab.es/~iblanes/beamer_gallery/index_by_theme.html
%
\mode<presentation>
{
  \usetheme{Madrid}      % or try Darmstadt, Madrid, Warsaw, ...
  \usecolortheme{seahorse} % or try albatross, beaver, crane, ...
  \usefonttheme{serif}  % or try serif, structurebold, ...
  \setbeamertemplate{navigation symbols}{}
  \setbeamertemplate{caption}[numbered]
  \usepackage{amsmath}
  \usepackage{tcolorbox}
  \usepackage[export]{adjustbox}
  \tcbuselibrary{most}
  \usepackage{arydshln}
 %\usepackage{enumitem}
%\usepackage{enumerate}
  %\usepackage[shortlabels]{enumitem}
} 


\definecolor{myblue}{RGB}{65,105,225} 
\definecolor{myorange}{RGB}{250,190,0}

\setbeamercolor{structure}{fg=white,bg=myorange}
\setbeamercolor*{palette primary}{fg=myblue,bg=myorange}
\setbeamercolor*{palette secondary}{fg=white,bg=myblue}
\setbeamercolor*{palette tertiary}{bg=myblue,fg=white}
\setbeamercolor*{palette quaternary}{fg=white,bg=myorange!50}

\setbeamercolor{frametitle}{fg=black!90!myblue}

\setbeamercolor{section in head/foot}{fg=white,bg=myblue}
\setbeamercolor{author in head/foot}{fg=black,bg=myorange}
\setbeamercolor{title in head/foot}{fg=white,bg=myblue}

\setbeamertemplate{navigation symbols}{}

\defbeamertemplate*{headline}{mytheme}
{%
  \begin{beamercolorbox}[ht=2.25ex,dp=3.75ex]{section in head/foot}
    \insertnavigation{\paperwidth}
  \end{beamercolorbox}%
}%

\defbeamertemplate*{footline}{mytheme}
{
  \leavevmode%
  \hbox{%
  \begin{beamercolorbox}[wd=.5\paperwidth,ht=2.25ex,dp=1ex,right]{author in head/foot}%
    \usebeamerfont{author in head/foot}\insertshortauthor\hspace*{2em}
  \end{beamercolorbox}%
  \begin{beamercolorbox}[wd=.5\paperwidth,ht=2.25ex,dp=1ex,left]{title in head/foot}%
    \usebeamerfont{title in head/foot}\hspace*{2em}\insertshortsubtitle\hspace*{2em}
    \insertframenumber{} / \inserttotalframenumber
  \end{beamercolorbox}}%
  \vskip0pt%
}



\usepackage[english]{babel}
\usepackage[utf8x]{inputenc}
\usepackage{xcolor}
\usepackage{listings}
\usepackage{pgf}  
\usepackage{textpos}
\usepackage{tabulary}
\usepackage{scrextend}
\usepackage{hyperref}
\usepackage{setspace}
\usepackage{rotating}
\lstset
{
    language=[LaTeX]TeX,
    breaklines=true,
    basicstyle=\tt\scriptsize,
    %commentstyle=\color{green}
    keywordstyle=\color{blue},
    %stringstyle=\color{black}
    identifierstyle=\color{magenta},
}

\usepackage{stackengine}
\def\Ruble{\stackengine{.67ex}{%
  \stackengine{.48ex}{\textsf{P}}{\rule{.8ex}{.12ex}\kern.6ex}{O}{r}{F}{F}{L}%
  }{\rule{.8ex}{.12ex}\kern.6ex}{O}{r}{F}{F}{L}\kern-.1ex}



%----------------------------------------------------------------------------------------
%	TITLE PAGE
%----------------------------------------------------------------------------------------
%\logo{\pgfbox[top]{\includegraphics[width=1.6cm]{logo-iu.png}}}
\title[L01]{Intr. alla data science e al pensiero computazionale \\ Lezione 2 e 3: Scrivere documenti con un linguaggio di programmazione -- \LaTeX{}} % The short title appears at the bottom of every slide, the full title is only on the title page

\author[{\tiny Giancarlo Succi}]{Giancarlo Succi \\ sulla base del lavoro originale John D. Lees-Miller \\ I dettagli nel seguito} % Your name
\institute[unibo] % Your institution as it will appear on the bottom of every slide, may be shorthand to save space

\date{} % Date, can be changed to a custom date
%\pgfdeclareimage[width=2.5cm]{logo}{logo-iu.png}
%\logo{\pgfuseimage{logo}{\vspace{10pt}}}

\setbeamertemplate{navigation symbols}{}
\AtBeginSection[]
{
        \begin{frame}<beamer>{Outline}
                \tableofcontents[currentsection]
        \end{frame}
}
\begin{document}
\begin{frame}
\titlepage % Print the title page as the first slide
\end{frame}

%=============================================

\addtobeamertemplate{frametitle}{}{%
\begin{textblock*}{10mm}(-0.01mm,-1cm)
\includegraphics[width=0.9cm]{unibo-logo.png}
\end{textblock*}}

%=============================================
\begin{frame}[label={L:Fonti}]
{\centerline{Fonti}}
\begin{itemize}
    \item Tutte queste diapositive provengono dal lavoro di John Lees-Miller presente sul sito \url{https://github.com/jdleesmiller/latex-course} e sono state tradotte e adattate dal docente del corso, essendo state messe a disposizione dall'autore con una licenza idonea. 
    \item Ulteriori fonti verranno dettagliate in seguito.
\end{itemize}
\end{frame}

\begin{frame}
{\centerline{\LaTeX{} $\ldots{}$ Perch\'{e}?}}
\begin{itemize}
\item Rende i documenti bellissimi!
\begin{itemize}
\item Anche quelli con formule e dati -- ideale per i professionisti dei dati e della transizione digitale
\end{itemize}
%
\item \`{E} stato creato da scienziati per scienziati.
\begin{itemize}
\item Ed \`{e} mantenuto da una comunit\`{a} vasta in tutto il mondo.
\end{itemize}
%
\item \`{E} molto potente e pu\`{o} anche essere esteso.
\begin{itemize}
\item Ci sono estensioni particolari per articoli, presentazioni, fogli elettronici, libri e ogni altra possibile necessit\`{a}
\end{itemize}
\end{itemize}
\end{frame}

%%%%%%%%%%%%%%%%%%%%%%%%%%%%%%%%%%%%%%%%%%%%%%%%%%%%%%%%%%%%%%%%%%%%%%%%%%%%%%%
%%%%%%%%%%%%%%%%%%%%%%%%%%%%%%%%%%%%%%%%%%%%%%%%%%%%%%%%%%%%%%%%%%%%%%%%%%%%%%%
%%%%%%%%%%%%%%%%%%%%%%%%%%%%%%%%%%%%%%%%%%%%%%%%%%%%%%%%%%%%%%%%%%%%%%%%%%%%%%%

\begin{frame}[fragile]{\centerline{L'idea di fondo}}
\begin{itemize}
\item Si scrivono i documenti focalizzandosi sul \texttt{testo}, sul suo significato senza preoccuparsi della formattazione
\item Si aggiungono \cmd{comandi} che descrivono la veste grafica e la struttura dei documenti
\item Il programma \texttt{latex} analizza il \texttt{testo} ed i   \cmd{comandi} e produce un documento cotto a puntino. 
\end{itemize}
\vskip 2ex
\begin{center}
\begin{minted}[frame=single]{latex}
La nebbia in Italia si trova \emph{soprattutto} a Milano.
\end{minted}
\vskip 2ex
\tikz\node[single arrow,fill=gray,font=\ttfamily\bfseries,%
  rotate=270,xshift=-1em]{latex};
\vskip 2ex
\fbox{La nebbia in Italia si trova \emph{soprattutto} a Milano.}
\end{center}
\end{frame}

%%%%%%%%%%%%%%%%%%%%%%%%%%%%%%%%%%%%%%%%%%%%%%%%%%%%%%%%%%%%%%%%%%%%%%%%%%%%%%%
\begin{frame}[fragile]{\centerline{Esempi di comandi}}
\begin{exampletwoup}
\begin{itemize}
\item Espresso
\item Cappuccino
\item Caff\`{e} latte
\end{itemize}
\end{exampletwoup}
\vskip 2ex
\begin{exampletwoup}
\begin{figure}
\includegraphics{A2022.IDSEPCLaTeX/gerbil.jpg}
\end{figure}
\end{exampletwoup}
\vskip 2ex
\begin{exampletwoup}
\begin{equation}
\alpha + \beta + 1
\end{equation}
\end{exampletwoup}

\tiny{Licenza per le immagini: \href{https://pixabay.com/en/animal-apple-attractive-beautiful-1239390/}{CC0}}
\end{frame}

%%%%%%%%%%%%%%%%%%%%%%%%%%%%%%%%%%%%%%%%%%%%%%%%%%%%%%%%%%%%%%%%%%%%%%%%%%%%%%%
%%%%%%%%%%%%%%%%%%%%%%%%%%%%%%%%%%%%%%%%%%%%%%%%%%%%%%%%%%%%%%%%%%%%%%%%%%%%%%%
%%%%%%%%%%%%%%%%%%%%%%%%%%%%%%%%%%%%%%%%%%%%%%%%%%%%%%%%%%%%%%%%%%%%%%%%%%%%%%%
\begin{frame}[fragile]{\centerline{L'approccio da seguire}}

\begin{itemize}
\item Bisogna innanzitutto focalizzarsi sul contenuto, ricordandosi Catone ``Rem tene, verba sequentur'' ... ``Rem tene, species sequetur''
\item I comandi vanno poi usati per specificare di che cosa si tratta e non come deve apparire.
\item Quindi si lascia \LaTeX{} a fare il suo lavoro.
\item La presentazione grafica pu\`{o} poi essere adattara configurando appropriatamente i comandi.
\end{itemize}
\end{frame}

%%%%%%%%%%%%%%%%%%%%%%%%%%%%%%%%%%%%%%%%%%%%%%%%%%%%%%%%%%%%%%%%%%%%%%%%%%%%%%%
%%%%%%%%%%%%%%%%%%%%%%%%%%%%%%%%%%%%%%%%%%%%%%%%%%%%%%%%%%%%%%%%%%%%%%%%%%%%%%%
%%%%%%%%%%%%%%%%%%%%%%%%%%%%%%%%%%%%%%%%%%%%%%%%%%%%%%%%%%%%%%%%%%%%%%%%%%%%%%%
%\section{The Basics}

%%%%%%%%%%%%%%%%%%%%%%%%%%%%%%%%%%%%%%%%%%%%%%%%%%%%%%%%%%%%%%%%%%%%%%%%%%%%%%%
%%%%%%%%%%%%%%%%%%%%%%%%%%%%%%%%%%%%%%%%%%%%%%%%%%%%%%%%%%%%%%%%%%%%%%%%%%%%%%%
%%%%%%%%%%%%%%%%%%%%%%%%%%%%%%%%%%%%%%%%%%%%%%%%%%%%%%%%%%%%%%%%%%%%%%%%%%%%%%%
\begin{frame}[fragile]{\centerline{Fiato alle trombe!}}
\begin{itemize}
\item Un documento \LaTeX{} minimo:
\inputminted[frame=single]{latex}{A2022.IDSEPCLaTeX/minimale.tex}
\item I comandi iniziano con un \emph{backslash} \keystrokebftt{\bs}.
\item Ogni documento inizia col comando \cmdbs{documentclass} che specifica la natura del documento stesso.
\item L'\emph{argomento} in parentesi graffe \keystrokebftt{\{} \keystrokebftt{\}} comunica a \LaTeX{} il tipo di documento che stiamo creando -- nel nostro caso un articolo, in inglese \bftt{article}.
\item Il simbolo di percentuale \keystrokebftt{\%} inizia una regione di testo che viene usata per fornire un commento per i lettori del testo \LaTeX{} e non del documento finale, in inglese si chiama \emph{comment} 
\begin{itemize}
\item \LaTeX{} ignora tutto il testo fino alla fine della linea.
\end{itemize}
\end{itemize}
\end{frame}

%%%%%%%%%%%%%%%%%%%%%%%%%%%%%%%%%%%%%%%%%%%%%%%%%%%%%%%%%%%%%%%%%%%%%%%%%%%%%%%
%%%%%%%%%%%%%%%%%%%%%%%%%%%%%%%%%%%%%%%%%%%%%%%%%%%%%%%%%%%%%%%%%%%%%%%%%%%%%%%
%%%%%%%%%%%%%%%%%%%%%%%%%%%%%%%%%%%%%%%%%%%%%%%%%%%%%%%%%%%%%%%%%%%%%%%%%%%%%%%
\begin{frame}[fragile]{\centerline{Usando \wllogo}}
\begin{itemize}
\item Overleaf \`{e} un sito web \textit{parzialmente} gratuito per scrivere documenti in \LaTeX.
\item Processa automaticamente il documento \LaTeX{} e visualizza il risultato.
\item In questo modo non occorre installare \LaTeX{} sul proprio calcolatore
\item Questo \`{e} il motivo del suo uso in questo corso
\item Gli svantaggi:
\begin{itemize}
\item occorre praticamente essere online nella versione gratuita
\item la versione gratuita ha limitazioni di dimensioni, collaboratori, ecc;
\end{itemize}
\end{itemize}

\end{frame}

\begin{frame}[fragile]{\centerline{Proviamo \wllogo}}
\begin{itemize}
\begin{center}
\fbox{\href{\wlnewdoc{minimale.tex}}{%
Cliccare qui per aprire un documento di esempio in \wllogo{}}}
\\[1ex]\scriptsize{}
\`{E} preferibile usare come browser  \href{http://www.google.com/chrome}{Google Chrome},  \href{http://www.mozilla.org/en-US/firefox/new/}{FireFox}, o \href{https://www.apple.com/safari/}{Safari}.
\end{center}
\vskip 2ex
\item A casa provate su \wllogo ~ gli esempi che stiamo analizzando in classe.
\item \textbf{La pratica \`{e} una parte essenziale per la comprensione dell'importanza di questo strumento di produzione di documenti!}
\end{itemize}
\end{frame}

%%%%%%%%%%%%%%%%%%%%%%%%%%%%%%%%%%%%%%%%%%%%%%%%%%%%%%%%%%%%%%%%%%%%%%%%%%%%%%%
%%%%%%%%%%%%%%%%%%%%%%%%%%%%%%%%%%%%%%%%%%%%%%%%%%%%%%%%%%%%%%%%%%%%%%%%%%%%%%%
%%%%%%%%%%%%%%%%%%%%%%%%%%%%%%%%%%%%%%%%%%%%%%%%%%%%%%%%%%%%%%%%%%%%%%%%%%%%%%%
\begin{frame}[fragile]{\centerline{Proviamo ad elaborare un testo}}
\small
\begin{itemize}
\item Scrivete il vostro testo tra \cmdbegin{document} e \cmdend{document}.
\item Nella maggior parte dei casi potete scrivere liberamente senza preoccuparvi in alcun modo della formattazione.
\begin{exampletwouptiny}
Le parole sono separate     da uno
o pi\`{u} spazi.

I paragrafi sono separati da una
o pi\`{u} linee bianche.

\end{exampletwouptiny}
\item Gli spazi e gli a capo singoli nel testo originario (detto anche \textit{testo sorgente}) sono di fatto ignorati e il testo \`{e} reso in quella che \`{e} ritenuta la forma ideale a prescindere da essi.
\begin{exampletwouptiny}
La nebbia      in Italia si trova
      soprattutto    a
Milano.
\end{exampletwouptiny}
\end{itemize}
\end{frame}

\begin{frame}[fragile]{\centerline{Le lettere accentate}}
\small
\begin{itemize}
\item Per le lettere accentate \ldots
\begin{exampletwouptiny}

Le lettere accentate sono scritte
in modo particolare, per via della
codifica
del testo e della loro grande
variet\`{a}.

In linea di massima, la sequenza \`{x}
mette l'accento grave su una
qualsivoglia lettera x,
\'{x} mette l'accento acuto,
\^{x} mette quello circonflesso.

La dieresi si ottiene con i doppi
apici \"{x} e la tilde ...
con la tilde \~{x}

\end{exampletwouptiny}
\end{itemize}
\end{frame}


%%%%%%%%%%%%%%%%%%%%%%%%%%%%%%%%%%%%%%%%%%%%%%%%%%%%%%%%%%%%%%%%%%%%%%%%%%%%%%%
%%%%%%%%%%%%%%%%%%%%%%%%%%%%%%%%%%%%%%%%%%%%%%%%%%%%%%%%%%%%%%%%%%%%%%%%%%%%%%%
%%%%%%%%%%%%%%%%%%%%%%%%%%%%%%%%%%%%%%%%%%%%%%%%%%%%%%%%%%%%%%%%%%%%%%%%%%%%%%%

\begin{frame}[fragile]{\centerline{Virgolette}}
\small
\begin{itemize}
\item Mettere una parola tra virgolette richiede attenzione:\\
si usano la virgoletta inverse \keystroke{\`{}} alla sinistra e l'apostrogo \keystroke{\'{}} alla destra.

\begin{exampletwouptiny}
Virgolette singole: `testo'.

Virgolette doppie: ``testo''.
\end{exampletwouptiny}

\end{itemize}
\end{frame}


\begin{frame}[fragile]{\centerline{Caratteri con un significato particolare}}
\small
\begin{itemize}

\item In \LaTeX ci sono  alcuni caratteri con un significato particolare:\\[1ex]
\begin{tabular}{cl}
\keystrokebftt{\%} & percentuale              \\
\keystrokebftt{\#} & cancelletto \\
\keystrokebftt{\&} & e commerciale                 \\
\keystrokebftt{\$} & dollaro               \\
\end{tabular}
\item Essi servono proprio per dare quei comandi che permettono di visualizzare il testo come desideriamo.
\item Se questi caratteri sono usati sic et simpliciter, otteniamo un errore. Se vogliamo usarli dobbiamo farli precedere dalla barretta inversa.
\item Questa operazione \`{e} detta generazione di una sequenza di fuga dal comando, in inglese \emph{escape}.
\begin{exampletwoup}
\$\%\&\#!
\end{exampletwoup}
\end{itemize}
\end{frame}

\begin{frame}[fragile]{\centerline{Caratteri con un significato \textbf{molto} particolare}}
\small
\begin{itemize}

\item In \LaTeX ci sono  alcuni caratteri con un significato \textbf{molto} particolare:\\[1ex]
\begin{tabular}{cl}
\keystrokebftt{\textbackslash} & barretta inversa              \\
\keystrokebftt{\texttildelow} & tilde bassa \\
\keystrokebftt{\textasciitilde} & tilde alta \\
\keystrokebftt{$\sim$} & tilde centrale \\
\end{tabular}
\item La doppia barretta inversa, infatti, si usa quando si vuole forzare un a capo.
\begin{exampletwoup}
\textbackslash ~ \`{e}
la barretta inversa.
\end{exampletwoup}
\item La tilde si usa quando si vuole forzare la presenza di uno spazio.
\begin{exampletwoup}
\texttildelow ~ \`{e}
la tilde bassa e \textasciitilde
~ \`{e} quella alta. Per quella
centrale bisogna usare le formule
matematiche $\sim$, di cui dopo.
\end{exampletwoup}
\end{itemize}
\end{frame}


%%%%%%%%%%%%%%%%%%%%%%%%%%%%%%%%%%%%%%%%%%%%%%%%%%%%%%%%%%%%%%%%%%%%%%%%%%%%%%%
%%%%%%%%%%%%%%%%%%%%%%%%%%%%%%%%%%%%%%%%%%%%%%%%%%%%%%%%%%%%%%%%%%%%%%%%%%%%%%%
%%%%%%%%%%%%%%%%%%%%%%%%%%%%%%%%%%%%%%%%%%%%%%%%%%%%%%%%%%%%%%%%%%%%%%%%%%%%%%%
\begin{frame}[fragile]{\centerline{Gestione degli errori}}
\begin{itemize}
\item La scrittura di un documento porta con s\'{e} la probabilit\`{a} molto alta di fare errori, tra i quali, errori nella scrittura dei comandi \LaTeX{}
\item \LaTeX{} prova a risolvere da solo gli errori, ma non \`{e} sempre in grado a gestire tali errori
\item Quando non ci riesce, si ferma con un messaggio che segnala la presenza dell'errore e bisogna risolvere tale errore prima di poter procedere
\item Per esempio se scriviamo \cmdbs{meph} invece di \cmdbs{emph} \LaTeX{} si ferma con il messaggio ``undefined control sequence'', in quanto ``meph'' non \`{e} un comando conosciuto
\item Il messaggio di errore alle volte \`{e} incomprensibile, per questo una ricerca su Google pu\`{o} essere molto utile
\item Ci sono parecchi siti dedicati alla risoluzione dei problemi di  \LaTeX{}, tra i quali la sezione di \href{https://tex.stackexchange.com}{stackexchange} dedicata a  \LaTeX{}
\end{itemize}
\end{frame}

\begin{frame}[fragile]{\centerline{Raccomandazioni per gli errori}}
\begin{itemize}
\item Non fatevi prendere dal panico.
\item L'errore non \`{e} segno di poca attenzione sul lavoro
\begin{itemize}
\item \normalsize ma un normalissimo, anche se indesiderato, effetto collaterale del lavoro
\item mia nonna diceva ``Chi fa falla, chi non fa \ldots farfalla''
\end{itemize}
\item La regola d'oro \`{e} di non lasciare errori irrisolti, ma, al contrario,
\begin{itemize}
\item \normalsize di affrontarli non appena appaiono.
\end{itemize}
\item Se ci sono poi errori multipli, \`{e} bene iniziare dal primo, \begin{itemize}
\item \normalsize infatti i successivi possono essere semplicemente causati da tale primo errore.
\end{itemize}
\end{itemize}
\end{frame}


%%%%%%%%%%%%%%%%%%%%%%%%%%%%%%%%%%%%%%%%%%%%%%%%%%%%%%%%%%%%%%%%%%%%%%%%%%%%%%%
%%%%%%%%%%%%%%%%%%%%%%%%%%%%%%%%%%%%%%%%%%%%%%%%%%%%%%%%%%%%%%%%%%%%%%%%%%%%%%%
%%%%%%%%%%%%%%%%%%%%%%%%%%%%%%%%%%%%%%%%%%%%%%%%%%%%%%%%%%%%%%%%%%%%%%%%%%%%%%%
\begin{frame}[fragile]{\centerline{Esercizio 1 (1/2)}}
\begin{itemize}
    \item Inserire in \LaTeX ~ il seguente testo preso da \hyperlink{https://it.wikipedia.org/wiki/Smithsonian_Agreement}{wikipedia}:
    \begin{itemize}
\item Nell'agosto 1971, infatti, il presidente statunitense Richard Nixon approv\`{o} la legge che sospendeva l'obbligo per la Federal Reserve di convertire dollari in oro al rapporto fisso di \$35 l'oncia, stabilito nel 1944 a Bretton Woods. Al contempo, fu introdotta una tassa del 10\% sulle importazioni negli Stati Uniti. Finiva così l'epoca dello standard oro-dollaro.

Tale decisione rischiava per\`{o} di provocare il caos nell'economia mondiale, che si trovava improvvisamente senza un sistema monetario internazionale. Fu cos\`{i} che nel dicembre dello stesso anno, i rappresentanti del Gruppo dei Dieci si riunirono a Washington, presso lo Smithsonian Institute. Ne nacque il cosiddetto Smithsonian Agreement, con il quale si decise una svalutazione del dollaro del 7,9\%
fissando un cambio di \$38 per oncia d'oro.
\end{itemize}
\item Originale: \url{https://it.wikipedia.org/wiki/Smithsonian_Agreement}
\end{itemize}
\end{frame}

\begin{frame}[fragile]{\centerline{Esercizio 1 (2/2)}}


\begin{center}
\fbox{\href{\wlnewdoc{minimale-esercizio-1.tex}}{%
Cliccare qui per aprire l'esercizio su \wllogo{}}}
\end{center}

\begin{itemize}
\item Suggerimento: controllare i caratteri con un significato particolare!
\item Completato l'esercizio,
\fbox{\href{\wlnewdoc{minimale-esercizio-1-soluzione.tex}}{%
qui c'\`{e} una soluzione}}.
\end{itemize}
\end{frame}

%%%%%%%%%%%%%%%%%%%%%%%%%%%%%%%%%%%%%%%%%%%%%%%%%%%%%%%%%%%%%%%%%%%%%%%%%%%%%%%
%%%%%%%%%%%%%%%%%%%%%%%%%%%%%%%%%%%%%%%%%%%%%%%%%%%%%%%%%%%%%%%%%%%%%%%%%%%%%%%
%%%%%%%%%%%%%%%%%%%%%%%%%%%%%%%%%%%%%%%%%%%%%%%%%%%%%%%%%%%%%%%%%%%%%%%%%%%%%%%
\begin{frame}[fragile]{\centerline{Formule matematiche -- Il \$}}
\begin{itemize}
\item Il simbolo di \keystrokebftt{\$} delimita testo da interpretare come formule matematiche.\\[1ex]
\begin{exampletwouptiny}
% non tanto bello:
Siano a e b due numeri interi
positivi, e sia c = a - b + 1.

% molto meglio:
Siano $a$ e $b$ due numeri interi
positivi, e sia $c = a - b + 1$.
\end{exampletwouptiny}
\item Il simbolo dollaro va usato sempre in coppia -- uno per iniziare e uno per concludere le formule matematiche
\item \LaTeX{} gestisce autonomamente lo spazio; ignora gli spazi inseriti manualmente.
\begin{exampletwouptiny}
Sia $y=mx+b$  \ldots

Sia $y = m x + b$  \ldots
\end{exampletwouptiny}
\end{itemize}
\end{frame}

%%%%%%%%%%%%%%%%%%%%%%%%%%%%%%%%%%%%%%%%%%%%%%%%%%%%%%%%%%%%%%%%%%%%%%%%%%%%%%%
%%%%%%%%%%%%%%%%%%%%%%%%%%%%%%%%%%%%%%%%%%%%%%%%%%%%%%%%%%%%%%%%%%%%%%%%%%%%%%%
%%%%%%%%%%%%%%%%%%%%%%%%%%%%%%%%%%%%%%%%%%%%%%%%%%%%%%%%%%%%%%%%%%%%%%%%%%%%%%%
\begin{frame}[fragile]{\centerline{Apici e pedici}}
\begin{itemize}
\item Usare la freccetta in alto \keystrokebftt{\^} per gli apici e la trattino basso  \keystrokebftt{\_} per i pedici.
\begin{exampletwouptiny}
$y = c_2 x^2 + c_1 x + c_0$
\end{exampletwouptiny}
\vskip 2ex

\item Se pi\`{u} di un carattere \`{e} in apice o pedice, occorre delimitare la sequenza in apice/pedice con le parentesi graffe \keystrokebftt{\{} \keystrokebftt{\}} .
\begin{exampletwouptiny}
$F_n = F_n-1 + F_n-2$     % attenzione!

$F_n = F_{n-1} + F_{n-2}$ % va bene!
\end{exampletwouptiny}
\vskip 2ex

\item Ci sono sequenze particolari per le lettere greche.
\begin{exampletwouptiny}
$\mu = A e^{Q/RT}$

$\Omega = \sum_{k=1}^{n} \omega_k$
\end{exampletwouptiny}
\end{itemize}
\end{frame}

%%%%%%%%%%%%%%%%%%%%%%%%%%%%%%%%%%%%%%%%%%%%%%%%%%%%%%%%%%%%%%%%%%%%%%%%%%%%%%%
%%%%%%%%%%%%%%%%%%%%%%%%%%%%%%%%%%%%%%%%%%%%%%%%%%%%%%%%%%%%%%%%%%%%%%%%%%%%%%%
%%%%%%%%%%%%%%%%%%%%%%%%%%%%%%%%%%%%%%%%%%%%%%%%%%%%%%%%%%%%%%%%%%%%%%%%%%%%%%%
\begin{frame}[fragile]{\centerline{Formule}}
\begin{itemize}
\item Se una formula \`{e} ampia, va presentata sulla una riga tutta per s\'{e} con la sequenza di comandi 
\cmdbegin{equation} e \cmdend{equation}. Ricordarsi che in inglese la parola ``formula'' si traduce spesso con ``equation''.\\[2ex]
\begin{exampletwouptiny}
La formula per la soluzione dell'equazione
di secondo grado \`{e} data da:
\begin{equation}
x = \frac{-b \pm \sqrt{b^2 - 4ac}}
         {2a}
\end{equation}
dove $a$, $b$ e $c$ sono \ldots
\end{exampletwouptiny}
\vskip 1em
{\scriptsize Attenzione: \LaTeX{} ignora gli spazi nelle formule matematiche ma non ignora le linee bianche, che \textbf{non} vanno inserite.}
\end{itemize}
\end{frame}

%%%%%%%%%%%%%%%%%%%%%%%%%%%%%%%%%%%%%%%%%%%%%%%%%%%%%%%%%%%%%%%%%%%%%%%%%%%%%%%
%%%%%%%%%%%%%%%%%%%%%%%%%%%%%%%%%%%%%%%%%%%%%%%%%%%%%%%%%%%%%%%%%%%%%%%%%%%%%%%
%%%%%%%%%%%%%%%%%%%%%%%%%%%%%%%%%%%%%%%%%%%%%%%%%%%%%%%%%%%%%%%%%%%%%%%%%%%%%%%
\begin{frame}[fragile]{\centerline{Environment ovvero Ambiente (1/3)}}
\begin{itemize}
\item Il termine \bftt{equation} messo come argomeno di una coppia \cmdbs{begin} e \cmdbs{end} definisce un ambiente, in inglese, un \emph{environment} --- uno spazio dove il comportamento di \LaTeX{} \`{e} particolare.
\item Un comando pu\`{o} produrre risultati diversi in ambienti diversi.
\begin{exampletwouptiny}
Si pu\`{o} scrivere
$ \Omega = \sum_{k=1}^{n} \omega_k $
nel testo oppure:
\begin{equation}
  \Omega = \sum_{k=1}^{n} \omega_k
\end{equation}
con risultati diversi.
\end{exampletwouptiny}
\vskip 2ex
\item Si noti che $\Sigma$ \`{e} pi\`{u} grande nell'ambiente \bftt{equation} environment, e che apici e pedici cambiano posizione, anche se il comando usato \`{e} lo stesso.

\end{itemize}
\end{frame}

%%%%%%%%%%%%%%%%%%%%%%%%%%%%%%%%%%%%%%%%%%%%%%%%%%%%%%%%%%%%%%%%%%%%%%%%%%%%%%%
%%%%%%%%%%%%%%%%%%%%%%%%%%%%%%%%%%%%%%%%%%%%%%%%%%%%%%%%%%%%%%%%%%%%%%%%%%%%%%%
%%%%%%%%%%%%%%%%%%%%%%%%%%%%%%%%%%%%%%%%%%%%%%%%%%%%%%%%%%%%%%%%%%%%%%%%%%%%%%%
\begin{frame}[fragile]{\centerline{Environment ovvero Ambiente (2/3)}}
\begin{itemize}
\item  In realt\`{a}, avremmo potuto scrivere invece di \bftt{\$...\$} come
\cmdbegin{math}\bftt{...}\cmdend{math} ovvero \bftt{\$...\$} sono una scorciatoia per definire un ambiente \bftt{math}.
\item La coppia \cmdbs{begin} e \cmdbs{end} \`{e} usata per creare qualsivoglia ambiente e ogni utente di \LaTeX{} pu\`{o} scrivere i propri ambienti per le proprie esigenze -- si veda ad esempio il sillabo del corso su \bftt{github}.
\vskip 2ex

\item Gli ambienti \bftt{itemize} e \bftt{enumerate} servono per definire lista.
\begin{exampletwouptiny}
\begin{itemize} % lista puntsata
\item Cornetto
\item Cappuccino
\end{itemize}

\begin{enumerate} % lista numerata
\item Cornetto
\item Cappuccino
\end{enumerate}
\end{exampletwouptiny}
\end{itemize}
\end{frame}

\begin{frame}[fragile]{\centerline{Environment ovvero Ambiente (3/3)}}
\begin{itemize}
\item  Ops ... la lista numerata non funziona.
\item Siamo in un ambiente particolare -- i lucidi, in cui il comportamento \`{e} ridefinito
\item Il numero c'\`{e} ma \ldots non si vede!
\item Per vederlo meglio occorre una ulteriore ridefinizione:
\vskip 2ex
\begin{exampletwouptiny}
\setbeamercolor{item projected}{bg=magenta}
\setbeamertemplate{enumerate items}[circle]
\begin{enumerate}
\item Cornetto
\item Cappuccino
\end{enumerate}
\end{exampletwouptiny}

\end{itemize}
\end{frame}


%%%%%%%%%%%%%%%%%%%%%%%%%%%%%%%%%%%%%%%%%%%%%%%%%%%%%%%%%%%%%%%%%%%%%%%%%%%%%%%
%%%%%%%%%%%%%%%%%%%%%%%%%%%%%%%%%%%%%%%%%%%%%%%%%%%%%%%%%%%%%%%%%%%%%%%%%%%%%%%
%%%%%%%%%%%%%%%%%%%%%%%%%%%%%%%%%%%%%%%%%%%%%%%%%%%%%%%%%%%%%%%%%%%%%%%%%%%%%%%
\begin{frame}[fragile]{\centerline{I package ovvero pacchetti}}

\begin{itemize}
\item Tutti i comandi che abbiamo visto finora sono presenti nella configurazione standard di 
\LaTeX.

\item \emph{Packages} o pacchetti sono biblioteche (in inglese, libraries) contenenti ulteriori comandi e ambienti. Ci sono letterlamente migliaia di pacchetti liberamente disponibili con licenza gratuita.

\item Per evitare una ovvia confusione, occorre caricare ogni pacchetto che vogliamo usare con il comando 
\cmdbs{usepackage} nel \emph{preambolo} del documento, in inglese \emph{preamble}, ovvero nella parte iniziale del file, prima dell'inizio del nostro testo.

\item Esempio: \bftt{amsmath} della American Mathematical Society.
\begin{minted}[fontsize=\small,frame=single]{latex}
\documentclass{article}
\usepackage{amsmath} % preambolo/preamble
\begin{document}
% ora possiamo usare qui i
% comandi di amsmath ...
\end{document}
\end{minted}
\end{itemize}
\end{frame}

%%%%%%%%%%%%%%%%%%%%%%%%%%%%%%%%%%%%%%%%%%%%%%%%%%%%%%%%%%%%%%%%%%%%%%%%%%%%%%%
%%%%%%%%%%%%%%%%%%%%%%%%%%%%%%%%%%%%%%%%%%%%%%%%%%%%%%%%%%%%%%%%%%%%%%%%%%%%%%%
%%%%%%%%%%%%%%%%%%%%%%%%%%%%%%%%%%%%%%%%%%%%%%%%%%%%%%%%%%%%%%%%%%%%%%%%%%%%%%%
\begin{frame}[fragile]{\centerline{Esempi con \bftt{amsmath} (1/2)}}
\begin{itemize}
\item \bftt{equation*} (``equation-star'') va usata per formule senza numero.
\begin{exampletwouptiny}
\begin{equation*}
  \Omega = \sum_{k=1}^{n} \omega_k
\end{equation*}
\end{exampletwouptiny}
\item \LaTeX{} tratta lettere adiacenti l'una all'altra come variabili moltiplicate tra loro -- non sempre quello che si vuole.  \bftt{amsmath} definisce i comandi per molti operatori matematici.
\begin{exampletwouptiny}
\begin{equation*} % brutto!
 min_{x,y} (1-x)^2 + 100(y-x^2)^2
\end{equation*}
\begin{equation*} % bello!
\min_{x,y}{(1-x)^2 + 100(y-x^2)^2}
\end{equation*}
\end{exampletwouptiny}
\item Si pu\`{o} anche usare \cmdbs{operatorname} per altri casi.
\begin{exampletwouptiny}
\begin{equation*}
\beta_i =
\frac{\operatorname{Cov}(R_i, R_m)}
     {\operatorname{Var}(R_m)}
\end{equation*}
\end{exampletwouptiny}
\end{itemize}
\end{frame}

%%%%%%%%%%%%%%%%%%%%%%%%%%%%%%%%%%%%%%%%%%%%%%%%%%%%%%%%%%%%%%%%%%%%%%%%%%%%%%%
%%%%%%%%%%%%%%%%%%%%%%%%%%%%%%%%%%%%%%%%%%%%%%%%%%%%%%%%%%%%%%%%%%%%%%%%%%%%%%%
%%%%%%%%%%%%%%%%%%%%%%%%%%%%%%%%%%%%%%%%%%%%%%%%%%%%%%%%%%%%%%%%%%%%%%%%%%%%%%%
\begin{frame}[fragile]{\centerline{Esempi con \bftt{amsmath} (2/2)}}
\begin{itemize}{\small
\item Se si vuole allineare una sequenza di espressioni al simbolo uguale:
\begin{align*}
(x+1)^3 &= (x+1)(x+1)(x+1) \\
        &= (x+1)(x^2 + 2x + 1) \\
        &= x^3 + 3x^2 + 3x + 1
\end{align*}
si pu\`{o} usare l'ambiente \bftt{align*}.

% for whatever reason, this doesn't play well with the twoup environment
\begin{minted}[fontsize=\small,frame=single]{latex}
\begin{align*}
(x+1)^3 &= (x+1)(x+1)(x+1) \\
        &= (x+1)(x^2 + 2x + 1) \\
        &= x^3 + 3x^2 + 3x + 1
\end{align*}
\end{minted}
\item La e commerciale \keystrokebftt{\&} separa la parte sinistra (prima del simbolo
$=$) dalla parte destra (dopo il $=$).
\item La doppia barra inversa \keystrokebftt{\bs}\keystrokebftt{\bs} definisce un a capo:  
}\end{itemize}
\end{frame}


%%%%%%%%%%%%%%%%%%%%%%%%%%%%%%%%%%%%%%%%%%%%%%%%%%%%%%%%%%%%%%%%%%%%%%%%%%%%%%%
%%%%%%%%%%%%%%%%%%%%%%%%%%%%%%%%%%%%%%%%%%%%%%%%%%%%%%%%%%%%%%%%%%%%%%%%%%%%%%%
%%%%%%%%%%%%%%%%%%%%%%%%%%%%%%%%%%%%%%%%%%%%%%%%%%%%%%%%%%%%%%%%%%%%%%%%%%%%%%%
\begin{frame}[fragile]{\centerline{Esercizio 2}}
\begin{itemize}
    \item Inserire in \LaTeX ~ il seguente testo:
\end{itemize}
\item Sia $X_1, X_2, \ldots, X_n$ una sequenza di variabili aleatorie indipendenti e distribuite identicamente con media $\operatorname{E}[X_i] = \mu$ e varianza
$\operatorname{Var}[X_i] = \sigma^2 < \infty$, e sia
\begin{equation*}
S_n = \frac{1}{n}\sum_{i=1}^{n} X_i
\end{equation*}
la media. Quando $n$ tende all'infinito, le variabili aleatorie
$\sqrt{n}(S_n - \mu)$ convergono verso una distribuzione normale $N(\mu, \sigma^2)$.
\vskip 2ex
\begin{center}
\fbox{\href{\wlnewdoc{minimale-esercizio-2.tex}}{%
Cliccare qui per aprire l'esercizio su \wllogo{}}}
\end{center}
\begin{itemize}
\item Suggerimento: il comando $\infty$ \`{e} \cmdbs{infty}.
\item Completato l'esercizio,
\fbox{\href{\wlnewdoc{minimale-esercizio-2-soluzione.tex}}{%
 qui c\`{e} una soluzione}}.
\end{itemize}
\end{frame}

%%%%%%%%%%%%%%%%%%%%%%%%%%%%%%%%%%%%%%%%%%%%%%%%%%%%%%%%%%%%%%%%%%%%%%%%%%%%%%%
%%%%%%%%%%%%%%%%%%%%%%%%%%%%%%%%%%%%%%%%%%%%%%%%%%%%%%%%%%%%%%%%%%%%%%%%%%%%%%%
%%%%%%%%%%%%%%%%%%%%%%%%%%%%%%%%%%%%%%%%%%%%%%%%%%%%%%%%%%%%%%%%%%%%%%%%%%%%%%%
\begin{frame}{\centerline{Facciamo il punto}}
\begin{itemize}
\item Fino ad ora abbiamo discusso \ldots
\begin{itemize}
\item Come scrivere un testo in \LaTeX.
\item L'uso di una variet\`{a} di comandi.
\item La gestione degli errori.
\item Le formule matematiche.
\item Il concetto di ambiente (\textit{environment}).
\item L'uso dei pacchetti (\textit{package}).
\end{itemize}
\item Ora ci occupiamo di come usare \LaTeX{} per scrivere documenti:
\begin{itemize}
\item strutturati i capitoli, sezioni, paragrafi e con 
\item riferimenti incrociati, figure, tabelli e riferimenti bibliografici.
\end{itemize}
\end{itemize}
\end{frame}

\begin{frame}[fragile]{\centerline{Titolo (\bftt{title}) e riassunto (\bftt{abstract})}}
\begin{itemize}{\small
\item Specificare nel preambolo di \LaTeX{} titolo (\cmdbs{title}) e autore/i (\cmdbs{author})
\item Poi usare il comando di generazione del titolo (\cmdbs{maketitle}) per crearlo nel documento.
\item Usa l'ambiente \bftt{abstract} per creare il riassunto.
}\end{itemize}
\begin{minipage}{0.55\linewidth}
\inputminted[fontsize=\scriptsize,frame=single,resetmargins]{latex}%
  {A2022.IDSEPCLaTeX/struttura-titolo.tex}
\end{minipage}
\begin{minipage}{0.35\linewidth}
\includegraphics[width=\textwidth,clip,trim=2.2in 7in 2.2in 2in]{A2022.IDSEPCLaTeX/struttura-titolo.pdf}
\end{minipage}
\end{frame}

%%%%%%%%%%%%%%%%%%%%%%%%%%%%%%%%%%%%%%%%%%%%%%%%%%%%%%%%%%%%%%%%%%%%%%%%%%%%%%%
%%%%%%%%%%%%%%%%%%%%%%%%%%%%%%%%%%%%%%%%%%%%%%%%%%%%%%%%%%%%%%%%%%%%%%%%%%%%%%%
%%%%%%%%%%%%%%%%%%%%%%%%%%%%%%%%%%%%%%%%%%%%%%%%%%%%%%%%%%%%%%%%%%%%%%%%%%%%%%%
\begin{frame}{\centerline{Sezioni (\bftt{section})}}
\begin{itemize}{\small
\item Basta usare \cmdbs{section},  \cmdbs{subsection} e \cmdbs{subsubsection}.
\item Provare a indovinare il significato di \cmdbs{section*} e \cmdbs{subsection*}...
}\end{itemize}
\begin{minipage}{0.55\linewidth}
\inputminted[fontsize=\scriptsize,frame=single,resetmargins]{latex}%
  {A2022.IDSEPCLaTeX/struttura-sezione.tex}
\end{minipage}
\begin{minipage}{0.35\linewidth}
\includegraphics[width=\textwidth,clip,trim=1.5in 6in 4in 1in]{A2022.IDSEPCLaTeX/struttura-sezione.pdf}
\end{minipage}
\end{frame}

%%%%%%%%%%%%%%%%%%%%%%%%%%%%%%%%%%%%%%%%%%%%%%%%%%%%%%%%%%%%%%%%%%%%%%%%%%%%%%%
%%%%%%%%%%%%%%%%%%%%%%%%%%%%%%%%%%%%%%%%%%%%%%%%%%%%%%%%%%%%%%%%%%%%%%%%%%%%%%%
%%%%%%%%%%%%%%%%%%%%%%%%%%%%%%%%%%%%%%%%%%%%%%%%%%%%%%%%%%%%%%%%%%%%%%%%%%%%%%%
\subsection{\centerline{Etichette (\cmdbs{label}) e riferimenti incrociati}}
\begin{frame}[fragile]{\insertsubsection}
\begin{itemize}{\small
\item Si usano i comandi \cmdbs{label} e \cmdbs{ref} .
\item Nel package \bftt{amsmath} c'\`{e} \cmdbs{eqref} per far riferimento a equazioni.
}\end{itemize}
\begin{minipage}{0.55\linewidth}
\inputminted[fontsize=\scriptsize,frame=single,resetmargins]{latex}%
  {A2022.IDSEPCLaTeX/struttura-riferimenti-incrociati.tex}
\end{minipage}
\begin{minipage}{0.35\linewidth}
\includegraphics[width=\textwidth,clip,trim=1.8in 6in 1.6in 1in]{A2022.IDSEPCLaTeX/struttura-riferimenti-incrociati.pdf}
\end{minipage}
\end{frame}

%%%%%%%%%%%%%%%%%%%%%%%%%%%%%%%%%%%%%%%%%%%%%%%%%%%%%%%%%%%%%%%%%%%%%%%%%%%%%%%
%%%%%%%%%%%%%%%%%%%%%%%%%%%%%%%%%%%%%%%%%%%%%%%%%%%%%%%%%%%%%%%%%%%%%%%%%%%%%%%
%%%%%%%%%%%%%%%%%%%%%%%%%%%%%%%%%%%%%%%%%%%%%%%%%%%%%%%%%%%%%%%%%%%%%%%%%%%%%%%
\begin{frame}[fragile]{\centerline{Esercizi sui documenti strutturati}}
\begin{itemize}
    \item 
Scrivi questo articolo in \LaTeX:
\footnote{Da \url{http://pdos.csail.mit.edu/scigen/}, un generatore randomico di testi, tradotto ed adattato dal docente.}

\vskip 4ex
\begin{center}
\fbox{\href{\fileuri/struttura-esercizio-soluzione.pdf}{%
Cliccare qui per visualizzare l'articolo.}}
\end{center}

\vskip 4ex
Formattare l'articolo come questo modello. Usare \cmdbs{ref} e \cmdbs{eqref} per evitare di scrivere in modo esplicito i numeri di sezione e di equazione.

\vskip 4ex
\begin{center}
\fbox{\href{\wlnewdoc{struttura-esercizio.tex}}{%
Cliccare qui per aprire l'esercizio in \wllogo{}}}
\end{center}
\end{itemize}
\vskip 2ex
\begin{itemize}
\item Completato l'esercizio,
\fbox{\href{\wlnewdoc{struttura-esercizio-soluzione.tex}}{%
qui c'\`{e} una soluzione}}.
\end{itemize}
\end{frame}


\begin{frame}[fragile]{\centerline{La grafica}}
\begin{itemize}
\item È necessario caricare il pacchetto \bftt{graphicx}, in quanto fornize il comando
\cmdbs{includegraphics}.
\item I formati gestiti sono JPEG, PNG e PDF (di solito).
\end{itemize}
\vskip 4ex
\begin{exampletwouptiny}
\includegraphics[
  width=0.5\textwidth]{A2022.IDSEPCLaTeX/gerbil}

\includegraphics[
  width=0.3\textwidth,
  angle=270]{gerbil}
\end{exampletwouptiny}

\tiny{Licenza per l'immagine: \href{https://pixabay.com/en/animal-apple-attractive-beautiful-1239390/}{CC0}}
\end{frame}

%%%%%%%%%%%%%%%%%%%%%%%%%%%%%%%%%%%%%%%%%%%%%%%%%%%%%%%%%%%%%%%%%%%%%%%%%%%%%%%
%%%%%%%%%%%%%%%%%%%%%%%%%%%%%%%%%%%%%%%%%%%%%%%%%%%%%%%%%%%%%%%%%%%%%%%%%%%%%%%
%%%%%%%%%%%%%%%%%%%%%%%%%%%%%%%%%%%%%%%%%%%%%%%%%%%%%%%%%%%%%%%%%%%%%%%%%%%%%%%
\begin{frame}[fragile]{Gli argomenti opzionali}
\begin{itemize}
\item Per gli argomenti opzionali usiamo le parentesi quadre \keystrokebftt{[} \keystrokebftt{]} invece delle parentesi graffe \keystrokebftt{\{} \keystrokebftt{\}}.
\item \cmdbs{includegraphics} accetta argomenti opzionali che permettono di modificare l'immagine una volta che \`{e} stata inclusa.

Per esempio, \bftt{width=0.5\cmdbs{textwidth}} fa in modo che la larghezza dell'immagine sia il 50\% della larghezza del testo  (\cmdbs{textwidth}).
\item \cmdbs{documentclass} pure accetta argomenti opzioni:
\mint{latex}|\documentclass[12pt,twocolumn]{article}|
rende il carattere del testo pi\`{u} grande (12pt) and puts e dispone il testo su due colonne.
\item Ulteriori informazioni su dove reperire queste informazioni sono provviste nel seguito.
\end{itemize}
\end{frame}

%%%%%%%%%%%%%%%%%%%%%%%%%%%%%%%%%%%%%%%%%%%%%%%%%%%%%%%%%%%%%%%%%%%%%%%%%%%%%%%
%%%%%%%%%%%%%%%%%%%%%%%%%%%%%%%%%%%%%%%%%%%%%%%%%%%%%%%%%%%%%%%%%%%%%%%%%%%%%%%
%%%%%%%%%%%%%%%%%%%%%%%%%%%%%%%%%%%%%%%%%%%%%%%%%%%%%%%%%%%%%%%%%%%%%%%%%%%%%%%
\begin{frame}{\centerline{Flottanti}}
\begin{itemize}
\item Quando si tratta di scegliere la posizione di una figura \~{e} meglio lasciare \LaTeX{} a decidere dove la posizione, cio\~{e} lasciarla ``flottante''.
\item Si pu\`{o} aggiungere una didascalia (in inglese \bftt{caption}) alla figura, che poi pu\`{o} essere utilizzata per un riferimento con
\cmdbs{ref}.
\end{itemize}
\begin{minipage}{0.55\linewidth}
\inputminted[fontsize=\scriptsize,frame=single,resetmargins]{latex}%
  {A2022.IDSEPCLaTeX/inserimento-figura.tex}
\end{minipage}
\begin{minipage}{0.35\linewidth}
\includegraphics[width=\textwidth,clip,trim=2in 5in 3in 1in]{A2022.IDSEPCLaTeX/inserimento-figura.pdf}
\end{minipage}

\tiny{Licenza per l'immagine: \href{https://pixabay.com/en/animal-apple-attractive-beautiful-1239390/}{CC0}}
\end{frame}

%%%%%%%%%%%%%%%%%%%%%%%%%%%%%%%%%%%%%%%%%%%%%%%%%%%%%%%%%%%%%%%%%%%%%%%%%%%%%%%
%%%%%%%%%%%%%%%%%%%%%%%%%%%%%%%%%%%%%%%%%%%%%%%%%%%%%%%%%%%%%%%%%%%%%%%%%%%%%%%
%%%%%%%%%%%%%%%%%%%%%%%%%%%%%%%%%%%%%%%%%%%%%%%%%%%%%%%%%%%%%%%%%%%%%%%%%%%%%%%
\subsection{Tables}
\begin{frame}[fragile]{\insertsubsection}
\begin{itemize}
\item Tables in \LaTeX{} take some getting used to.
\item Use the \bftt{tabular} environment from the \bftt{tabularx} package.
\item The argument specifies column alignment --- \textbf{l}eft, \textbf{r}ight, \textbf{r}ight.
\begin{exampletwouptiny}
\begin{tabular}{lrr}
Item   & Qty & Unit \$ \\
Widget & 1   & 199.99  \\
Gadget & 2   & 399.99  \\
Cable  & 3   & 19.99   \\
\end{tabular}
\end{exampletwouptiny}
\item It also specifies vertical lines; use \cmdbs{hline} for horizontal lines.
\begin{exampletwouptiny}
\begin{tabular}{|l|r|r|} \hline
Item   & Qty & Unit \$ \\\hline
Widget & 1   & 199.99  \\
Gadget & 2   & 399.99  \\
Cable  & 3   & 19.99   \\\hline
\end{tabular}
\end{exampletwouptiny}
\item Use an ampersand \keystrokebftt{\&} to separate columns and a double backslash \keystrokebftt{\bs}\keystrokebftt{\bs} to start a new row (like in the \bftt{align*} environment that we saw in part 1).
\end{itemize}
\end{frame}

%%%%%%%%%%%%%%%%%%%%%%%%%%%%%%%%%%%%%%%%%%%%%%%%%%%%%%%%%%%%%%%%%%%%%%%%%%%%%%%
%%%%%%%%%%%%%%%%%%%%%%%%%%%%%%%%%%%%%%%%%%%%%%%%%%%%%%%%%%%%%%%%%%%%%%%%%%%%%%%
%%%%%%%%%%%%%%%%%%%%%%%%%%%%%%%%%%%%%%%%%%%%%%%%%%%%%%%%%%%%%%%%%%%%%%%%%%%%%%%
\addtocontents{toc}{\newpage}
%\section{Bibliographies}

%%%%%%%%%%%%%%%%%%%%%%%%%%%%%%%%%%%%%%%%%%%%%%%%%%%%%%%%%%%%%%%%%%%%%%%%%%%%%%%
%%%%%%%%%%%%%%%%%%%%%%%%%%%%%%%%%%%%%%%%%%%%%%%%%%%%%%%%%%%%%%%%%%%%%%%%%%%%%%%
%%%%%%%%%%%%%%%%%%%%%%%%%%%%%%%%%%%%%%%%%%%%%%%%%%%%%%%%%%%%%%%%%%%%%%%%%%%%%%%
\begin{frame}{Outline}
\begin{multicols}{2}
\tableofcontents[currentsection]
\end{multicols}
\end{frame}

%%%%%%%%%%%%%%%%%%%%%%%%%%%%%%%%%%%%%%%%%%%%%%%%%%%%%%%%%%%%%%%%%%%%%%%%%%%%%%%
%%%%%%%%%%%%%%%%%%%%%%%%%%%%%%%%%%%%%%%%%%%%%%%%%%%%%%%%%%%%%%%%%%%%%%%%%%%%%%%
%%%%%%%%%%%%%%%%%%%%%%%%%%%%%%%%%%%%%%%%%%%%%%%%%%%%%%%%%%%%%%%%%%%%%%%%%%%%%%%
\subsection{bib\TeX}
\begin{frame}[fragile]{\insertsubsection{} 1}
\begin{itemize}
\item Put your references in a \bftt{.bib} file in `bibtex' database format:
\inputminted[fontsize=\scriptsize,frame=single]{latex}{bib-example.bib}
\item Most reference managers can export to bibtex format.
\end{itemize}
\end{frame}

%%%%%%%%%%%%%%%%%%%%%%%%%%%%%%%%%%%%%%%%%%%%%%%%%%%%%%%%%%%%%%%%%%%%%%%%%%%%%%%
%%%%%%%%%%%%%%%%%%%%%%%%%%%%%%%%%%%%%%%%%%%%%%%%%%%%%%%%%%%%%%%%%%%%%%%%%%%%%%%
%%%%%%%%%%%%%%%%%%%%%%%%%%%%%%%%%%%%%%%%%%%%%%%%%%%%%%%%%%%%%%%%%%%%%%%%%%%%%%%
\begin{frame}[fragile]{\insertsubsection{} 2}
\begin{itemize}
\item Each entry in the \bftt{.bib} file has a \emph{key} that you can use to
reference it in the document. For example, \bftt{Jacobson1999Towards} is the key for this article:
\begin{minted}[fontsize=\small,frame=single]{latex}
@Article{Jacobson1999Towards,
  author = {Van Jacobson},
  ...
}
\end{minted}
\item It's a good idea to use a key based on the name, year and title.
\item \LaTeX{} can automatically format your in-text citations and generate a
list of references; it knows most standard styles, and you can design your own.
\end{itemize}
\end{frame}

%%%%%%%%%%%%%%%%%%%%%%%%%%%%%%%%%%%%%%%%%%%%%%%%%%%%%%%%%%%%%%%%%%%%%%%%%%%%%%%
%%%%%%%%%%%%%%%%%%%%%%%%%%%%%%%%%%%%%%%%%%%%%%%%%%%%%%%%%%%%%%%%%%%%%%%%%%%%%%%
%%%%%%%%%%%%%%%%%%%%%%%%%%%%%%%%%%%%%%%%%%%%%%%%%%%%%%%%%%%%%%%%%%%%%%%%%%%%%%%
\begin{frame}[fragile]{\insertsubsection{} 3}
\begin{itemize}
\item Use the \bftt{natbib} package\footnote{There is a new package with more
  features named \bftt{biblatex} but most of the articles templates still use
  \bftt{natbib}.} with \cmdbs{citet} and \cmdbs{citep}.
\item Reference \cmdbs{bibliography} at the end, and specify a \cmdbs{bibliographystyle}.
\end{itemize}
\begin{minipage}{0.55\linewidth}
\inputminted[fontsize=\scriptsize,frame=single,resetmargins]{latex}%
  {bib-example.tex}
\end{minipage}
\begin{minipage}{0.35\linewidth}
\includegraphics[width=\textwidth,clip,trim=1.8in 5in 1.8in 1in]{bib-example.pdf}
\end{minipage}
\end{frame}

%%%%%%%%%%%%%%%%%%%%%%%%%%%%%%%%%%%%%%%%%%%%%%%%%%%%%%%%%%%%%%%%%%%%%%%%%%%%%%%
%%%%%%%%%%%%%%%%%%%%%%%%%%%%%%%%%%%%%%%%%%%%%%%%%%%%%%%%%%%%%%%%%%%%%%%%%%%%%%%
%%%%%%%%%%%%%%%%%%%%%%%%%%%%%%%%%%%%%%%%%%%%%%%%%%%%%%%%%%%%%%%%%%%%%%%%%%%%%%%
\subsection{Exercise}
\begin{frame}[fragile]{Exercise: Putting it All Together}

Add an image and a bibliography to the paper from the previous exercise.

\begin{enumerate}
\item Download these example files to your computer.

\begin{center}
\fbox{\href{\fileuri/gerbil.jpg?dl=1}{Click to download example image}}

\fbox{\href{\fileuri/bib-exercise.bib?dl=1}{Click to download example bib file}}
\end{center}

\item Upload them to Overleaf (use the project menu).

\end{enumerate}
\end{frame}

%%%%%%%%%%%%%%%%%%%%%%%%%%%%%%%%%%%%%%%%%%%%%%%%%%%%%%%%%%%%%%%%%%%%%%%%%%%%%%%
%%%%%%%%%%%%%%%%%%%%%%%%%%%%%%%%%%%%%%%%%%%%%%%%%%%%%%%%%%%%%%%%%%%%%%%%%%%%%%%
%%%%%%%%%%%%%%%%%%%%%%%%%%%%%%%%%%%%%%%%%%%%%%%%%%%%%%%%%%%%%%%%%%%%%%%%%%%%%%%
%\section{What's Next?}

%%%%%%%%%%%%%%%%%%%%%%%%%%%%%%%%%%%%%%%%%%%%%%%%%%%%%%%%%%%%%%%%%%%%%%%%%%%%%%%
%%%%%%%%%%%%%%%%%%%%%%%%%%%%%%%%%%%%%%%%%%%%%%%%%%%%%%%%%%%%%%%%%%%%%%%%%%%%%%%
%%%%%%%%%%%%%%%%%%%%%%%%%%%%%%%%%%%%%%%%%%%%%%%%%%%%%%%%%%%%%%%%%%%%%%%%%%%%%%%
\begin{frame}{Outline}
\begin{multicols}{2}
\tableofcontents[currentsection]
\end{multicols}
\end{frame}

%%%%%%%%%%%%%%%%%%%%%%%%%%%%%%%%%%%%%%%%%%%%%%%%%%%%%%%%%%%%%%%%%%%%%%%%%%%%%%%
%%%%%%%%%%%%%%%%%%%%%%%%%%%%%%%%%%%%%%%%%%%%%%%%%%%%%%%%%%%%%%%%%%%%%%%%%%%%%%%
%%%%%%%%%%%%%%%%%%%%%%%%%%%%%%%%%%%%%%%%%%%%%%%%%%%%%%%%%%%%%%%%%%%%%%%%%%%%%%%
\subsection{More Neat Things}
\begin{frame}[fragile]{\insertsubsection}
\begin{itemize}
\item Add the \cmdbs{tableofcontents} command to generate a table of contents
from the \cmdbs{section} commands.

\item Change the \cmdbs{documentclass} to
\mint{latex}!\documentclass{scrartcl}!
or
\mint{latex}!\documentclass[12pt]{IEEEtran}!

\item Define your own command for a complicated equation:
\begin{exampletwouptiny}
\newcommand{\rperf}{%
  \rho_{\text{perf}}}
$$
\rperf = {\bf c}'{\bf X} + \varepsilon
$$
\end{exampletwouptiny}
\end{itemize}
\end{frame}

%%%%%%%%%%%%%%%%%%%%%%%%%%%%%%%%%%%%%%%%%%%%%%%%%%%%%%%%%%%%%%%%%%%%%%%%%%%%%%%
%%%%%%%%%%%%%%%%%%%%%%%%%%%%%%%%%%%%%%%%%%%%%%%%%%%%%%%%%%%%%%%%%%%%%%%%%%%%%%%
%%%%%%%%%%%%%%%%%%%%%%%%%%%%%%%%%%%%%%%%%%%%%%%%%%%%%%%%%%%%%%%%%%%%%%%%%%%%%%%
\subsection{More Neat Packages}
\begin{frame}{\insertsubsection}
\begin{itemize}
\item \bftt{beamer}: for presentations (like this one!)
\item \bftt{todonotes}: comments and TODO management
\item \bftt{tikz}: make amazing graphics
\item \bftt{pgfplots}: create graphs in \LaTeX
\item \bftt{listings}: source code printer for \LaTeX
\item \bftt{spreadtab}: create spreadsheets in \LaTeX
\item \bftt{gchords}, \bftt{guitar}: guitar chords and tabulature
\item \bftt{cwpuzzle}: crossword puzzles
\end{itemize}
See \url{https://www.overleaf.com/latex/examples} and \url{http://texample.net}
for examples of (most of) these packages.
\end{frame}

%%%%%%%%%%%%%%%%%%%%%%%%%%%%%%%%%%%%%%%%%%%%%%%%%%%%%%%%%%%%%%%%%%%%%%%%%%%%%%%
%%%%%%%%%%%%%%%%%%%%%%%%%%%%%%%%%%%%%%%%%%%%%%%%%%%%%%%%%%%%%%%%%%%%%%%%%%%%%%%
%%%%%%%%%%%%%%%%%%%%%%%%%%%%%%%%%%%%%%%%%%%%%%%%%%%%%%%%%%%%%%%%%%%%%%%%%%%%%%%
\subsection{Installing \LaTeX{}}
\begin{frame}{\insertsubsection}
\begin{itemize}
\item To run \LaTeX{} on your own computer, you'll want to use a \LaTeX{}
\emph{distribution}. A distribution includes a \bftt{latex} program
and (typically) several thousand packages.
\begin{itemize}
\item On Windows: \href{http://miktex.org/}{Mik\TeX} or \href{http://tug.org/texlive/}{\TeX Live}
\item On Linux: \href{http://tug.org/texlive/}{\TeX Live}
\item On Mac: \href{http://tug.org/mactex/}{Mac\TeX}
\end{itemize}
\item You'll also want a text editor with \LaTeX{} support. See \url{http://en.wikipedia.org/wiki/Comparison_of_TeX_editors} for a list of (many) options.
\item You'll also have to know more about how \bftt{latex} and its related tools
work --- see the resources on the next slide.
\end{itemize}
\end{frame}

%%%%%%%%%%%%%%%%%%%%%%%%%%%%%%%%%%%%%%%%%%%%%%%%%%%%%%%%%%%%%%%%%%%%%%%%%%%%%%%
%%%%%%%%%%%%%%%%%%%%%%%%%%%%%%%%%%%%%%%%%%%%%%%%%%%%%%%%%%%%%%%%%%%%%%%%%%%%%%%
%%%%%%%%%%%%%%%%%%%%%%%%%%%%%%%%%%%%%%%%%%%%%%%%%%%%%%%%%%%%%%%%%%%%%%%%%%%%%%%
\subsection{Online Resources}
\begin{frame}{\insertsubsection}
\begin{itemize}
\item \href{https://www.overleaf.com/learn}{The Overleaf Learn Wiki} ---
hosts these slides, more tutorials and reference material
\item \href{http://en.wikibooks.org/wiki/LaTeX}{The \LaTeX{} Wikibook} ---
excellent tutorials and reference material.
\item \href{http://tex.stackexchange.com/}{\TeX{} Stack Exchange} --- ask
questions and get excellent answers incredibly quickly
\item \href{http://www.latex-community.org/}{\LaTeX{} Community} --- a large
online forum
\item \href{http://ctan.org/}{Comprehensive \TeX{} Archive Network (CTAN)} ---
over four thousand packages plus documentation
\item Google will usually get you to one of the above.
\end{itemize}
\end{frame}

%%%%%%%%%%%%%%%%%%%%%%%%%%%%%%%%%%%%%%%%%%%%%%%%%%%%%%%%%%%%%%%%%%%%%%%%%%%%%%%
%%%%%%%%%%%%%%%%%%%%%%%%%%%%%%%%%%%%%%%%%%%%%%%%%%%%%%%%%%%%%%%%%%%%%%%%%%%%%%%
%%%%%%%%%%%%%%%%%%%%%%%%%%%%%%%%%%%%%%%%%%%%%%%%%%%%%%%%%%%%%%%%%%%%%%%%%%%%%%%
\begin{frame}
\begin{center}
Thanks, and happy \TeX{}ing!
\end{center}
\end{frame}

\end{document}

% -- latex understands words, sentences and paragraphs

Words are separated by one or more spaces.  Paragraphs are separated by
one or more blank lines.  The output is not affected by adding extra
spaces or extra blank lines to the input file.

Double quotes are typed like this: ``quoted text''.
Single quotes are typed like this: `single-quoted text'.

Emphasized text is typed like this: \emph{this is emphasized}.
Bold       text is typed like this: \textbf{this is bold}.

-- Adding structure to your document

\section{Hello}

\subsection{World}

\subsection{Foo}

\subsubsection*{Stuff} % star form

\subsubsection*{Results}

-- Labels and cross-references

\label{sec:intro}
\label{sec:method}
\ref{sec:method}

--> maybe introduce the prettyref package here.

-- Mathematics

Inline mathematics: $x + y < 7$.

'Displayed' mathematics:
\begin{equation}
\end{equation}

\begin{equation*}
\end{equation*}

\begin{align}
\end{align}

-- Figures

- Need the graphicx package.

- here we can start introducing options

\includegraphics[width=\textwidth]{}

- where do you find out about these options? --> link to the Wikibook

-- Floating Figures

\begin{figure}
\includegraphics{...}
\caption{\label{}Here is a caption.}
\end{figure}

-- Tables

- not the nicest part of LaTeX

\usepackage{tabularx}

\begin{tabular}{llr}
Item & Quantity & Price (\$) & Amount
Widget & 1 &
\end{tabular}

Bonus points: check out the fp package and the spreadtab package.

-- Document Classes

a .cls file

article

some journal templates come with one

-- Bibliographies



-- For Typesetting Geeks

- dashes: -, --, ---

- ellipsis.

- controlling spaces: ~, \ , \,, \@

- spacing after periods (et al., etc.)

- Nested quotation marks: ``\,`
\vskip 2ex
\item Use the \emph{star form} to display an equation without a number.
\begin{exampletwouptiny}
\begin{equation*}
F(x) = \int_{a}^{x}{f(t) dt}
\end{equation*}
\end{exampletwouptiny}

\begin{itemize}
\item \bftt{equation} and \bftt{equation*} are called \emph{environments}.
\begin{itemize}
  \item The \cmdbs{begin} and \cmdbs{end} commands define the environment.
  \item The \cmd{\$} also starts and ends an environment.
  \item Some commands are defined only within certain environments.
  \item Some commands behave differently in different environments.
\end{itemize}
\end{itemize}
\end{block}
\begin{center}
\fbox{\href{http://ctan.org/}{The Comprehensive \TeX Archive Network (CTAN)}}
\end{center}

%%%%%%%%%%%%%%%%%%%%%%%%%%%%%%%%%%%%%%%%%%%%%%%%%%%%%%%%%%%%%%%%%%%%%%%%%%%%%%%
%%%%%%%%%%%%%%%%%%%%%%%%%%%%%%%%%%%%%%%%%%%%%%%%%%%%%%%%%%%%%%%%%%%%%%%%%%%%%%%
%%%%%%%%%%%%%%%%%%%%%%%%%%%%%%%%%%%%%%%%%%%%%%%%%%%%%%%%%%%%%%%%%%%%%%%%%%%%%%%
\subsection{Typography tweaks}
\begin{frame}{\insertsubsection}
\begin{tabular}{lll}
& character name & used mainly for \ldots \\\hline
\bftt{\bs} & backslash                 & commands, tables \\
\bftt{\{}  & open brace                & commands \\
\bftt{\}}  & close brace               & commands \\
\bftt{\%}  & percent sign              & comments \\
\bftt{\#}  & hash (pound / sharp) sign & custom commands \\
\bftt{\$}  & dollar sign               & equations \\
\bftt{\_}  & underscore                & equations (subscripts) \\
\bftt{\^}  & caret                     & equations (superscripts) \\
\bftt{\&}  & ampersand                 & tables \\
\bftt{\~}  & tilde                     & spacing \\
\end{tabular}
\end{frame}

%\item We've used several environments:
%\vskip 1ex
%{\scriptsize
%\begin{tabular}{ll}
%\cmdbs{begin}\bftt{\{document\}}\ldots\cmdbs{end}\bftt{\{document\}} &
%  document environment \\
%\cmdbs{begin}\bftt{\{itemize\}}\ldots\cmdbs{end}\bftt{\{itemize\}} &
%  itemized list environment \\
%\bftt{\$\ldots\$}     & \emph{in-text} math environment \\
%\bftt{\$\$\ldots\$\$} & \emph{displayed} math environment \\
%\cmdbs{begin}\bftt{\{equation\}}\ldots\cmdbs{end}\bftt{\{equation\}} &
%  displayed math environment w/ number
%\end{tabular}
%}

\end{document}
