\documentclass{beamer}
\input{A2022.IDSEPCLaTeX/preambolo}

%
% Choose how your presentation looks.
% For more themes, color themes and font themes, see:
% http://deic.uab.es/~iblanes/beamer_gallery/index_by_theme.html
%
\mode<presentation>
{
  \usetheme{Madrid}      % or try Darmstadt, Madrid, Warsaw, ...
  \usecolortheme{seahorse} % or try albatross, beaver, crane, ...
  \usefonttheme{serif}  % or try serif, structurebold, ...
  \setbeamertemplate{navigation symbols}{}
  \setbeamertemplate{caption}[numbered]
  \usepackage{amsmath}
  \usepackage{tcolorbox}
  \usepackage[export]{adjustbox}
  \tcbuselibrary{most}
  \usepackage{arydshln}
 %\usepackage{enumitem}
%\usepackage{enumerate}
  %\usepackage[shortlabels]{enumitem}
} 


\definecolor{myblue}{RGB}{65,105,225} 
\definecolor{myorange}{RGB}{250,190,0}

\setbeamercolor{structure}{fg=white,bg=myorange}
\setbeamercolor*{palette primary}{fg=myblue,bg=myorange}
\setbeamercolor*{palette secondary}{fg=white,bg=myblue}
\setbeamercolor*{palette tertiary}{bg=myblue,fg=white}
\setbeamercolor*{palette quaternary}{fg=white,bg=myorange!50}

\setbeamercolor{frametitle}{fg=black!90!myblue}

\setbeamercolor{section in head/foot}{fg=white,bg=myblue}
\setbeamercolor{author in head/foot}{fg=black,bg=myorange}
\setbeamercolor{title in head/foot}{fg=white,bg=myblue}

\setbeamertemplate{navigation symbols}{}

\defbeamertemplate*{headline}{mytheme}
{%
  \begin{beamercolorbox}[ht=2.25ex,dp=3.75ex]{section in head/foot}
    \insertnavigation{\paperwidth}
  \end{beamercolorbox}%
}%

\defbeamertemplate*{footline}{mytheme}
{
  \leavevmode%
  \hbox{%
  \begin{beamercolorbox}[wd=.5\paperwidth,ht=2.25ex,dp=1ex,right]{author in head/foot}%
    \usebeamerfont{author in head/foot}\insertshortauthor\hspace*{2em}
  \end{beamercolorbox}%
  \begin{beamercolorbox}[wd=.5\paperwidth,ht=2.25ex,dp=1ex,left]{title in head/foot}%
    \usebeamerfont{title in head/foot}\hspace*{2em}\insertshortsubtitle\hspace*{2em}
    \insertframenumber{} / \inserttotalframenumber
  \end{beamercolorbox}}%
  \vskip0pt%
}



\usepackage[english]{babel}
\usepackage[utf8x]{inputenc}
\usepackage{xcolor}
\usepackage{listings}
\usepackage{pgf}  
\usepackage{textpos}
\usepackage{tabulary}
\usepackage{scrextend}
\usepackage{hyperref}
\usepackage{setspace}
\usepackage{rotating}
\lstset
{
    language=[LaTeX]TeX,
    breaklines=true,
    basicstyle=\tt\scriptsize,
    %commentstyle=\color{green}
    keywordstyle=\color{blue},
    %stringstyle=\color{black}
    identifierstyle=\color{magenta},
}

\usepackage{stackengine}
\def\Ruble{\stackengine{.67ex}{%
  \stackengine{.48ex}{\textsf{P}}{\rule{.8ex}{.12ex}\kern.6ex}{O}{r}{F}{F}{L}%
  }{\rule{.8ex}{.12ex}\kern.6ex}{O}{r}{F}{F}{L}\kern-.1ex}



%----------------------------------------------------------------------------------------
%	TITLE PAGE
%----------------------------------------------------------------------------------------
%\logo{\pgfbox[top]{\includegraphics[width=1.6cm]{logo-iu.png}}}
\title[L01]{Intr. alla data science e al pensiero computazionale \\ Lezione 2 e 3: Scrivere documenti con un linguaggio di programmazione -- \LaTeX{}} % The short title appears at the bottom of every slide, the full title is only on the title page

\author[{\tiny Giancarlo Succi}]{Giancarlo Succi \\ sulla base del lavoro originale John D. Lees-Miller \\ I dettagli nel seguito} % Your name
\institute[unibo] % Your institution as it will appear on the bottom of every slide, may be shorthand to save space

\date{} % Date, can be changed to a custom date
%\pgfdeclareimage[width=2.5cm]{logo}{logo-iu.png}
%\logo{\pgfuseimage{logo}{\vspace{10pt}}}

\setbeamertemplate{navigation symbols}{}
\AtBeginSection[]
{
        \begin{frame}<beamer>{Outline}
                \tableofcontents[currentsection]
        \end{frame}
}
\begin{document}
\begin{frame}
\titlepage % Print the title page as the first slide
\end{frame}

%=============================================

\addtobeamertemplate{frametitle}{}{%
\begin{textblock*}{10mm}(-0.01mm,-1cm)
\includegraphics[width=0.9cm]{unibo-logo.png}
\end{textblock*}}

%=============================================
\begin{frame}[label={L:Fonti}]
{\centerline{Fonti}}
\begin{itemize}
    \item Tutte queste diapositive provengono dal lavoro di John Lees-Miller presente sul sito \url{https://github.com/jdleesmiller/latex-course} e sono state tradotte e adattate dal docente del corso, essendo state messe a disposizione dall'autore con una licenza idonea. 
    \item Ulteriori fonti verranno dettagliate in seguito.
\end{itemize}
\end{frame}

\begin{frame}
{\centerline{\LaTeX{} $\ldots{}$ Perch\'{e}?}}
\begin{itemize}
\item Rende i documenti bellissimi!
\begin{itemize}
\item Anche quelli con formule e dati -- ideale per i professionisti dei dati e della transizione digitale
\end{itemize}
%
\item \`{E} stato creato da scienziati per scienziati.
\begin{itemize}
\item Ed \`{e} mantenuto da una comunit\`{a} vasta in tutto il mondo.
\end{itemize}
%
\item \`{E} molto potente e pu\`{o} anche essere esteso.
\begin{itemize}
\item Ci sono estensioni particolari per articoli, presentazioni, fogli elettronici, libri e ogni altra possibile necessit\`{a}
\end{itemize}
\end{itemize}
\end{frame}

%%%%%%%%%%%%%%%%%%%%%%%%%%%%%%%%%%%%%%%%%%%%%%%%%%%%%%%%%%%%%%%%%%%%%%%%%%%%%%%
%%%%%%%%%%%%%%%%%%%%%%%%%%%%%%%%%%%%%%%%%%%%%%%%%%%%%%%%%%%%%%%%%%%%%%%%%%%%%%%
%%%%%%%%%%%%%%%%%%%%%%%%%%%%%%%%%%%%%%%%%%%%%%%%%%%%%%%%%%%%%%%%%%%%%%%%%%%%%%%

\begin{frame}[fragile]{\centerline{L'idea di fondo}}
\begin{itemize}
\item Si scrivono i documenti focalizzandosi sul \texttt{testo}, sul suo significato senza preoccuparsi della formattazione
\item Si aggiungono \cmd{comandi} che descrivono la veste grafica e la struttura dei documenti
\item Il programma \texttt{latex} analizza il \texttt{testo} ed i   \cmd{comandi} e produce un documento cotto a puntino. 
\end{itemize}
\vskip 2ex
\begin{center}
\begin{minted}[frame=single]{latex}
La nebbia in Italia si trova \emph{soprattutto} a Milano.
\end{minted}
\vskip 2ex
\tikz\node[single arrow,fill=gray,font=\ttfamily\bfseries,%
  rotate=270,xshift=-1em]{latex};
\vskip 2ex
\fbox{La nebbia in Italia si trova \emph{soprattutto} a Milano.}
\end{center}
\end{frame}

%%%%%%%%%%%%%%%%%%%%%%%%%%%%%%%%%%%%%%%%%%%%%%%%%%%%%%%%%%%%%%%%%%%%%%%%%%%%%%%
\begin{frame}[fragile]{\centerline{Esempi di comandi}}
\begin{exampletwoup}
\begin{itemize}
\item Espresso
\item Cappuccino
\item Caff\`{e} latte
\end{itemize}
\end{exampletwoup}
\vskip 2ex
\begin{exampletwoup}
\begin{figure}
\includegraphics{A2022.IDSEPCLaTeX/gerbil.jpg}
\end{figure}
\end{exampletwoup}
\vskip 2ex
\begin{exampletwoup}
\begin{equation}
\alpha + \beta + 1
\end{equation}
\end{exampletwoup}

\tiny{Licenza per le immagini: \href{https://pixabay.com/en/animal-apple-attractive-beautiful-1239390/}{CC0}}
\end{frame}

%%%%%%%%%%%%%%%%%%%%%%%%%%%%%%%%%%%%%%%%%%%%%%%%%%%%%%%%%%%%%%%%%%%%%%%%%%%%%%%
%%%%%%%%%%%%%%%%%%%%%%%%%%%%%%%%%%%%%%%%%%%%%%%%%%%%%%%%%%%%%%%%%%%%%%%%%%%%%%%
%%%%%%%%%%%%%%%%%%%%%%%%%%%%%%%%%%%%%%%%%%%%%%%%%%%%%%%%%%%%%%%%%%%%%%%%%%%%%%%
\begin{frame}[fragile]{\centerline{L'approccio da seguire}}

\begin{itemize}
\item Bisogna innanzitutto focalizzarsi sul contenuto, ricordandosi Catone ``Rem tene, verba sequentur'' ... ``Rem tene, species sequetur''
\item I comandi vanno poi usati per specificare di che cosa si tratta e non come deve apparire.
\item Quindi si lascia \LaTeX{} a fare il suo lavoro.
\item La presentazione grafica pu\`{o} poi essere adattara configurando appropriatamente i comandi.
\end{itemize}
\end{frame}

%%%%%%%%%%%%%%%%%%%%%%%%%%%%%%%%%%%%%%%%%%%%%%%%%%%%%%%%%%%%%%%%%%%%%%%%%%%%%%%
%%%%%%%%%%%%%%%%%%%%%%%%%%%%%%%%%%%%%%%%%%%%%%%%%%%%%%%%%%%%%%%%%%%%%%%%%%%%%%%
%%%%%%%%%%%%%%%%%%%%%%%%%%%%%%%%%%%%%%%%%%%%%%%%%%%%%%%%%%%%%%%%%%%%%%%%%%%%%%%
%\section{The Basics}

%%%%%%%%%%%%%%%%%%%%%%%%%%%%%%%%%%%%%%%%%%%%%%%%%%%%%%%%%%%%%%%%%%%%%%%%%%%%%%%
%%%%%%%%%%%%%%%%%%%%%%%%%%%%%%%%%%%%%%%%%%%%%%%%%%%%%%%%%%%%%%%%%%%%%%%%%%%%%%%
%%%%%%%%%%%%%%%%%%%%%%%%%%%%%%%%%%%%%%%%%%%%%%%%%%%%%%%%%%%%%%%%%%%%%%%%%%%%%%%
\begin{frame}[fragile]{\centerline{Fiato alle trombe!}}
\begin{itemize}
\item Un documento \LaTeX{} minimo:
\inputminted[frame=single]{latex}{A2022.IDSEPCLaTeX/minimale.tex}
\item I comandi iniziano con un \emph{backslash} \keystrokebftt{\bs}.
\item Ogni documento inizia col comando \cmdbs{documentclass} che specifica la natura del documento stesso.
\item L'\emph{argomento} in parentesi graffe \keystrokebftt{\{} \keystrokebftt{\}} comunica a \LaTeX{} il tipo di documento che stiamo creando -- nel nostro caso un articolo, in inglese \bftt{article}.
\item Il simbolo di percentuale \keystrokebftt{\%} inizia una regione di testo che viene usata per fornire un commento per i lettori del testo \LaTeX{} e non del documento finale, in inglese si chiama \emph{comment} 
\begin{itemize}
\item \LaTeX{} ignora tutto il testo fino alla fine della linea.
\end{itemize}
\end{itemize}
\end{frame}

%%%%%%%%%%%%%%%%%%%%%%%%%%%%%%%%%%%%%%%%%%%%%%%%%%%%%%%%%%%%%%%%%%%%%%%%%%%%%%%
%%%%%%%%%%%%%%%%%%%%%%%%%%%%%%%%%%%%%%%%%%%%%%%%%%%%%%%%%%%%%%%%%%%%%%%%%%%%%%%
%%%%%%%%%%%%%%%%%%%%%%%%%%%%%%%%%%%%%%%%%%%%%%%%%%%%%%%%%%%%%%%%%%%%%%%%%%%%%%%
\begin{frame}[fragile]{\centerline{Usando \wllogo}}
\begin{itemize}
\item Overleaf \`{e} un sito web \textit{parzialmente} gratuito per scrivere documenti in \LaTeX.
\item Processa automaticamente il documento \LaTeX{} e visualizza il risultato.
\item In questo modo non occorre installare \LaTeX{} sul proprio calcolatore
\item Questo \`{e} il motivo del suo uso in questo corso
\item Gli svantaggi:
\begin{itemize}
\item occorre praticamente essere online nella versione gratuita
\item la versione gratuita ha limitazioni di dimensioni, collaboratori, ecc;
\end{itemize}
\end{itemize}

\end{frame}

\begin{frame}[fragile]{\centerline{Proviamo \wllogo}}
\begin{itemize}
\begin{center}
\fbox{\href{\wlnewdoc{minimale.tex}}{%
Cliccare qui per aprire un documento di esempio in \wllogo{}}}
\\[1ex]\scriptsize{}
\`{E} preferibile usare come browser  \href{http://www.google.com/chrome}{Google Chrome},  \href{http://www.mozilla.org/en-US/firefox/new/}{FireFox}, o \href{https://www.apple.com/safari/}{Safari}.
\end{center}
\vskip 2ex
\item A casa provate su \wllogo ~ gli esempi che stiamo analizzando in classe.
\item \textbf{La pratica \`{e} una parte essenziale per la comprensione dell'importanza di questo strumento di produzione di documenti!}
\end{itemize}
\end{frame}

%%%%%%%%%%%%%%%%%%%%%%%%%%%%%%%%%%%%%%%%%%%%%%%%%%%%%%%%%%%%%%%%%%%%%%%%%%%%%%%
%%%%%%%%%%%%%%%%%%%%%%%%%%%%%%%%%%%%%%%%%%%%%%%%%%%%%%%%%%%%%%%%%%%%%%%%%%%%%%%
%%%%%%%%%%%%%%%%%%%%%%%%%%%%%%%%%%%%%%%%%%%%%%%%%%%%%%%%%%%%%%%%%%%%%%%%%%%%%%%
\begin{frame}[fragile]{\centerline{Proviamo ad elaborare un testo}}
\small
\begin{itemize}
\item Scrivete il vostro testo tra \cmdbegin{document} e \cmdend{document}.
\item Nella maggior parte dei casi potete scrivere liberamente senza preoccuparvi in alcun modo della formattazione.
\begin{exampletwouptiny}
Le parole sono separate     da uno
o pi\`{u} spazi.

I paragrafi sono separati da una
o pi\`{u} linee bianche.

\end{exampletwouptiny}
\item Gli spazi e gli a capo singoli nel testo originario (detto anche \textit{testo sorgente}) sono di fatto ignorati e il testo \`{e} reso in quella che \`{e} ritenuta la forma ideale a prescindere da essi.
\begin{exampletwouptiny}
La nebbia      in Italia si trova
      soprattutto    a
Milano.
\end{exampletwouptiny}
\end{itemize}
\end{frame}

\begin{frame}[fragile]{\centerline{Le lettere accentate}}
\small
\begin{itemize}
\item Per le lettere accentate \ldots
\begin{exampletwouptiny}

Le lettere accentate sono scritte
in modo particolare, per via della
codifica
del testo e della loro grande
variet\`{a}.

In linea di massima, la sequenza \`{x}
mette l'accento grave su una
qualsivoglia lettera x,
\'{x} mette l'accento acuto,
\^{x} mette quello circonflesso.

La dieresi si ottiene con i doppi
apici \"{x} e la tilde ...
con la tilde \~{x}

\end{exampletwouptiny}
\end{itemize}
\end{frame}


%%%%%%%%%%%%%%%%%%%%%%%%%%%%%%%%%%%%%%%%%%%%%%%%%%%%%%%%%%%%%%%%%%%%%%%%%%%%%%%
%%%%%%%%%%%%%%%%%%%%%%%%%%%%%%%%%%%%%%%%%%%%%%%%%%%%%%%%%%%%%%%%%%%%%%%%%%%%%%%
%%%%%%%%%%%%%%%%%%%%%%%%%%%%%%%%%%%%%%%%%%%%%%%%%%%%%%%%%%%%%%%%%%%%%%%%%%%%%%%

\begin{frame}[fragile]{\centerline{Virgolette}}
\small
\begin{itemize}
\item Mettere una parola tra virgolette richiede attenzione:\\
si usano la virgoletta inverse \keystroke{\`{}} alla sinistra e l'apostrogo \keystroke{\'{}} alla destra.
\begin{exampletwouptiny}
Virgolette singole: `testo'.

Virgolette doppie: ``testo''.
\end{exampletwouptiny}

\end{itemize}
\end{frame}


\begin{frame}[fragile]{\centerline{Caratteri con un significato particolare}}
\small
\begin{itemize}

\item In \LaTeX ci sono  alcuni caratteri con un significato particolare:\\[1ex]
\begin{tabular}{cl}
\keystrokebftt{\%} & percentuale              \\
\keystrokebftt{\#} & cancelletto \\
\keystrokebftt{\&} & e commerciale                 \\
\keystrokebftt{\$} & dollaro               \\
\end{tabular}
\item Essi servono proprio per dare quei comandi che permettono di visualizzare il testo come desideriamo.
\item Se questi caratteri sono usati sic et simpliciter, otteniamo un errore. Se vogliamo usarli dobbiamo farli precedere dalla barretta inversa.
\item Questa operazione \`{e} detta generazione di una sequenza di fuga dal comando, in inglese \emph{escape}.
\begin{exampletwoup}
\$\%\&\#!
\end{exampletwoup}
\end{itemize}
\end{frame}

\begin{frame}[fragile]{\centerline{Caratteri con un significato \textbf{molto} particolare}}
\small
\begin{itemize}

\item In \LaTeX ci sono  alcuni caratteri con un significato \textbf{molto} particolare:\\[1ex]
\begin{tabular}{cl}
\keystrokebftt{\textbackslash} & barretta inversa              \\
\keystrokebftt{\texttildelow} & tilde bassa \\
\keystrokebftt{\textasciitilde} & tilde alta \\
\keystrokebftt{$\sim$} & tilde centrale \\
\end{tabular}
\item La doppia barretta inversa, infatti, si usa quando si vuole forzare un a capo.
\begin{exampletwoup}
\textbackslash ~ \`{e}
la barretta inversa.
\end{exampletwoup}
\item La tilde si usa quando si vuole forzare la presenza di uno spazio.
\begin{exampletwoup}
\texttildelow ~ \`{e}
la tilde bassa e \textasciitilde
~ \`{e} quella alta. Per quella
centrale bisogna usare le formule
matematiche $\sim$, di cui dopo.
\end{exampletwoup}
\end{itemize}
\end{frame}


%%%%%%%%%%%%%%%%%%%%%%%%%%%%%%%%%%%%%%%%%%%%%%%%%%%%%%%%%%%%%%%%%%%%%%%%%%%%%%%
%%%%%%%%%%%%%%%%%%%%%%%%%%%%%%%%%%%%%%%%%%%%%%%%%%%%%%%%%%%%%%%%%%%%%%%%%%%%%%%
%%%%%%%%%%%%%%%%%%%%%%%%%%%%%%%%%%%%%%%%%%%%%%%%%%%%%%%%%%%%%%%%%%%%%%%%%%%%%%%
\begin{frame}[fragile]{\centerline{Gestione degli errori}}
\begin{itemize}
\item La scrittura di un documento porta con s\'{e} la probabilit\`{a} molto alta di fare errori, tra i quali, errori nella scrittura dei comandi \LaTeX{}
\item \LaTeX{} prova a risolvere da solo gli errori, ma non \`{e} sempre in grado a gestire tali errori
\item Quando non ci riesce, si ferma con un messaggio che segnala la presenza dell'errore e bisogna risolvere tale errore prima di poter procedere
\item Per esempio se scriviamo \cmdbs{meph} invece di \cmdbs{emph} \LaTeX{} si ferma con il messaggio ``undefined control sequence'', in quanto ``meph'' non \`{e} un comando conosciuto
\item Il messaggio di errore alle volte \`{e} incomprensibile, per questo una ricerca su Google pu\`{o} essere molto utile
\item Ci sono parecchi siti dedicati alla risoluzione dei problemi di  \LaTeX{}, tra i quali la sezione di \href{https://tex.stackexchange.com}{stackexchange} dedicata a  \LaTeX{}
\end{itemize}
\end{frame}

\begin{frame}[fragile]{\centerline{Raccomandazioni per gli errori}}
\begin{itemize}
\item Non fatevi prendere dal panico.
\item L'errore non \`{e} segno di poca attenzione sul lavoro
\begin{itemize}
\item \normalsize ma un normalissimo, anche se indesiderato, effetto collaterale del lavoro
\item mia nonna diceva ``Chi fa falla, chi non fa \ldots farfalla''
\end{itemize}
\item La regola d'oro \`{e} di non lasciare errori irrisolti, ma, al contrario,
\begin{itemize}
\item \normalsize di affrontarli non appena appaiono.
\end{itemize}
\item Se ci sono poi errori multipli, \`{e} bene iniziare dal primo, \begin{itemize}
\item \normalsize infatti i successivi possono essere semplicemente causati da tale primo errore.
\end{itemize}
\end{itemize}
\end{frame}


%%%%%%%%%%%%%%%%%%%%%%%%%%%%%%%%%%%%%%%%%%%%%%%%%%%%%%%%%%%%%%%%%%%%%%%%%%%%%%%
%%%%%%%%%%%%%%%%%%%%%%%%%%%%%%%%%%%%%%%%%%%%%%%%%%%%%%%%%%%%%%%%%%%%%%%%%%%%%%%
%%%%%%%%%%%%%%%%%%%%%%%%%%%%%%%%%%%%%%%%%%%%%%%%%%%%%%%%%%%%%%%%%%%%%%%%%%%%%%%
\begin{frame}[fragile]{\centerline{Esercizio 1 (1/2)}}
\begin{itemize}
    \item Inserire in \LaTeX ~ il seguente testo preso da \hyperlink{https://it.wikipedia.org/wiki/Smithsonian_Agreement}{wikipedia}:
    \begin{itemize}
\item Nell'agosto 1971, infatti, il presidente statunitense Richard Nixon approv\`{o} la legge che sospendeva l'obbligo per la Federal Reserve di convertire dollari in oro al rapporto fisso di \$35 l'oncia, stabilito nel 1944 a Bretton Woods. Al contempo, fu introdotta una tassa del 10\% sulle importazioni negli Stati Uniti. Finiva così l'epoca dello standard oro-dollaro.

Tale decisione rischiava per\`{o} di provocare il caos nell'economia mondiale, che si trovava improvvisamente senza un sistema monetario internazionale. Fu cos\`{i} che nel dicembre dello stesso anno, i rappresentanti del Gruppo dei Dieci si riunirono a Washington, presso lo Smithsonian Institute. Ne nacque il cosiddetto Smithsonian Agreement, con il quale si decise una svalutazione del dollaro del 7,9\%
fissando un cambio di \$38 per oncia d'oro.
\end{itemize}
\item Originale: \url{https://it.wikipedia.org/wiki/Smithsonian_Agreement}
\end{itemize}
\end{frame}

\begin{frame}[fragile]{\centerline{Esercizio 1 (2/2)}}


\begin{center}
\fbox{\href{\wlnewdoc{minimale-esercizio-1.tex}}{%
Cliccare qui per aprire l'esercizio su \wllogo{}}}
\end{center}

\begin{itemize}
\item Hint: watch out for characters with special meanings!
\item Once you've tried,
\fbox{\href{\wlnewdoc{minimale-esercizio-1-soluzione.tex}}{%
click here to see my solution}}.
\end{itemize}
\end{frame}

%%%%%%%%%%%%%%%%%%%%%%%%%%%%%%%%%%%%%%%%%%%%%%%%%%%%%%%%%%%%%%%%%%%%%%%%%%%%%%%
%%%%%%%%%%%%%%%%%%%%%%%%%%%%%%%%%%%%%%%%%%%%%%%%%%%%%%%%%%%%%%%%%%%%%%%%%%%%%%%
%%%%%%%%%%%%%%%%%%%%%%%%%%%%%%%%%%%%%%%%%%%%%%%%%%%%%%%%%%%%%%%%%%%%%%%%%%%%%%%
\subsection{Typesetting Mathematics}
\begin{frame}[fragile]{\insertsubsection{}: Dollar Signs}
\begin{itemize}
\item Why are dollar signs \keystrokebftt{\$} special? We use them to mark mathematics in text.\\[1ex]
\begin{exampletwouptiny}
% not so good:
Let a and b be distinct positive
integers, and let c = a - b + 1.

% much better:
Let $a$ and $b$ be distinct positive
integers, and let $c = a - b + 1$.
\end{exampletwouptiny}
\item Always use dollar signs in pairs --- one to begin the mathematics, and one
to end it.
\item \LaTeX{} handles spacing automatically; it ignores your spaces.
\begin{exampletwouptiny}
Let $y=mx+b$ be \ldots

Let $y = m x + b$ be \ldots
\end{exampletwouptiny}
\end{itemize}
\end{frame}

%%%%%%%%%%%%%%%%%%%%%%%%%%%%%%%%%%%%%%%%%%%%%%%%%%%%%%%%%%%%%%%%%%%%%%%%%%%%%%%
%%%%%%%%%%%%%%%%%%%%%%%%%%%%%%%%%%%%%%%%%%%%%%%%%%%%%%%%%%%%%%%%%%%%%%%%%%%%%%%
%%%%%%%%%%%%%%%%%%%%%%%%%%%%%%%%%%%%%%%%%%%%%%%%%%%%%%%%%%%%%%%%%%%%%%%%%%%%%%%
\begin{frame}[fragile]{\insertsubsection{}: Notation}
\begin{itemize}
\item Use caret \keystrokebftt{\^} for superscripts and underscore \keystrokebftt{\_} for subscripts.
\begin{exampletwouptiny}
$y = c_2 x^2 + c_1 x + c_0$
\end{exampletwouptiny}
\vskip 2ex

\item Use curly braces \keystrokebftt{\{} \keystrokebftt{\}} to group
superscripts and subscripts.
\begin{exampletwouptiny}
$F_n = F_n-1 + F_n-2$     % oops!

$F_n = F_{n-1} + F_{n-2}$ % ok!
\end{exampletwouptiny}
\vskip 2ex

\item There are commands for Greek letters and common notation.
\begin{exampletwouptiny}
$\mu = A e^{Q/RT}$

$\Omega = \sum_{k=1}^{n} \omega_k$
\end{exampletwouptiny}
\end{itemize}
\end{frame}

%%%%%%%%%%%%%%%%%%%%%%%%%%%%%%%%%%%%%%%%%%%%%%%%%%%%%%%%%%%%%%%%%%%%%%%%%%%%%%%
%%%%%%%%%%%%%%%%%%%%%%%%%%%%%%%%%%%%%%%%%%%%%%%%%%%%%%%%%%%%%%%%%%%%%%%%%%%%%%%
%%%%%%%%%%%%%%%%%%%%%%%%%%%%%%%%%%%%%%%%%%%%%%%%%%%%%%%%%%%%%%%%%%%%%%%%%%%%%%%
\begin{frame}[fragile]{\insertsubsection{}: Displayed Equations}
\begin{itemize}
\item If it's big and scary, \emph{display} it on its own line using
\cmdbegin{equation} and \cmdend{equation}.\\[2ex]
\begin{exampletwouptiny}
The roots of a quadratic equation
are given by
\begin{equation}
x = \frac{-b \pm \sqrt{b^2 - 4ac}}
         {2a}
\end{equation}
where $a$, $b$ and $c$ are \ldots
\end{exampletwouptiny}
\vskip 1em
{\scriptsize Caution: \LaTeX{} mostly ignores your spaces in mathematics, but it
can't handle blank lines in equations --- don't put blank lines in your
mathematics.}
\end{itemize}
\end{frame}

%%%%%%%%%%%%%%%%%%%%%%%%%%%%%%%%%%%%%%%%%%%%%%%%%%%%%%%%%%%%%%%%%%%%%%%%%%%%%%%
%%%%%%%%%%%%%%%%%%%%%%%%%%%%%%%%%%%%%%%%%%%%%%%%%%%%%%%%%%%%%%%%%%%%%%%%%%%%%%%
%%%%%%%%%%%%%%%%%%%%%%%%%%%%%%%%%%%%%%%%%%%%%%%%%%%%%%%%%%%%%%%%%%%%%%%%%%%%%%%
\begin{frame}[fragile]{Interlude: Environments}
\begin{itemize}
\item \bftt{equation} is an \emph{environment} --- a context.
\item A command can produce different output in different contexts.
\begin{exampletwouptiny}
We can write
$ \Omega = \sum_{k=1}^{n} \omega_k $
in text, or we can write
\begin{equation}
  \Omega = \sum_{k=1}^{n} \omega_k
\end{equation}
to display it.
\end{exampletwouptiny}
\vskip 2ex
\item Note how the $\Sigma$ is bigger in the \bftt{equation} environment, and
how the subscripts and superscripts change position, even though we used the
same commands.
\vskip 1em
{\scriptsize In fact, we could have written \bftt{\$...\$} as
\cmdbegin{math}\bftt{...}\cmdend{math}.}
\end{itemize}
\end{frame}

%%%%%%%%%%%%%%%%%%%%%%%%%%%%%%%%%%%%%%%%%%%%%%%%%%%%%%%%%%%%%%%%%%%%%%%%%%%%%%%
%%%%%%%%%%%%%%%%%%%%%%%%%%%%%%%%%%%%%%%%%%%%%%%%%%%%%%%%%%%%%%%%%%%%%%%%%%%%%%%
%%%%%%%%%%%%%%%%%%%%%%%%%%%%%%%%%%%%%%%%%%%%%%%%%%%%%%%%%%%%%%%%%%%%%%%%%%%%%%%
\begin{frame}[fragile]{Interlude: Environments}
\begin{itemize}
\item The \cmdbs{begin} and \cmdbs{end} commands are used to create many
different environments.
\vskip 2ex

\item The \bftt{itemize} and \bftt{enumerate} environments generate lists.
\begin{exampletwouptiny}
\begin{itemize} % for bullet points
\item Biscuits
\item Tea
\end{itemize}

\begin{enumerate} % for numbers
\item Biscuits
\item Tea
\end{enumerate}
\end{exampletwouptiny}
\end{itemize}
\end{frame}

%%%%%%%%%%%%%%%%%%%%%%%%%%%%%%%%%%%%%%%%%%%%%%%%%%%%%%%%%%%%%%%%%%%%%%%%%%%%%%%
%%%%%%%%%%%%%%%%%%%%%%%%%%%%%%%%%%%%%%%%%%%%%%%%%%%%%%%%%%%%%%%%%%%%%%%%%%%%%%%
%%%%%%%%%%%%%%%%%%%%%%%%%%%%%%%%%%%%%%%%%%%%%%%%%%%%%%%%%%%%%%%%%%%%%%%%%%%%%%%
\begin{frame}[fragile]{Interlude: Packages}

\begin{itemize}
\item All of the commands and environments we've used so far are built into
\LaTeX.

\item \emph{Packages} are libraries of extra commands and environments. There
are thousands of freely available packages.

\item We have to load each of the packages we want to use with a
\cmdbs{usepackage} command in the \emph{preamble}.

\item Example: \bftt{amsmath} from the American Mathematical Society.
\begin{minted}[fontsize=\small,frame=single]{latex}
\documentclass{article}
\usepackage{amsmath} % preamble
\begin{document}
% now we can use commands from amsmath here...
\end{document}
\end{minted}
\end{itemize}
\end{frame}

%%%%%%%%%%%%%%%%%%%%%%%%%%%%%%%%%%%%%%%%%%%%%%%%%%%%%%%%%%%%%%%%%%%%%%%%%%%%%%%
%%%%%%%%%%%%%%%%%%%%%%%%%%%%%%%%%%%%%%%%%%%%%%%%%%%%%%%%%%%%%%%%%%%%%%%%%%%%%%%
%%%%%%%%%%%%%%%%%%%%%%%%%%%%%%%%%%%%%%%%%%%%%%%%%%%%%%%%%%%%%%%%%%%%%%%%%%%%%%%
\begin{frame}[fragile]{\insertsubsection{}: Examples with \bftt{amsmath}}
\begin{itemize}
\item Use \bftt{equation*} (``equation-star'') for unnumbered equations.
\begin{exampletwouptiny}
\begin{equation*}
  \Omega = \sum_{k=1}^{n} \omega_k
\end{equation*}
\end{exampletwouptiny}
\item \LaTeX{} treats adjacent letters as variables multiplied together, which
is not always what you want. \bftt{amsmath} defines commands for many common
mathematical operators.
\begin{exampletwouptiny}
\begin{equation*} % bad!
 min_{x,y} (1-x)^2 + 100(y-x^2)^2
\end{equation*}
\begin{equation*} % good!
\min_{x,y}{(1-x)^2 + 100(y-x^2)^2}
\end{equation*}
\end{exampletwouptiny}
\item You can use \cmdbs{operatorname} for others.
\begin{exampletwouptiny}
\begin{equation*}
\beta_i =
\frac{\operatorname{Cov}(R_i, R_m)}
     {\operatorname{Var}(R_m)}
\end{equation*}
\end{exampletwouptiny}
\end{itemize}
\end{frame}

%%%%%%%%%%%%%%%%%%%%%%%%%%%%%%%%%%%%%%%%%%%%%%%%%%%%%%%%%%%%%%%%%%%%%%%%%%%%%%%
%%%%%%%%%%%%%%%%%%%%%%%%%%%%%%%%%%%%%%%%%%%%%%%%%%%%%%%%%%%%%%%%%%%%%%%%%%%%%%%
%%%%%%%%%%%%%%%%%%%%%%%%%%%%%%%%%%%%%%%%%%%%%%%%%%%%%%%%%%%%%%%%%%%%%%%%%%%%%%%
\begin{frame}[fragile]{\insertsubsection{}: Examples with \bftt{amsmath}}
\begin{itemize}{\small
\item Align a sequence of equations at the equals sign
\begin{align*}
(x+1)^3 &= (x+1)(x+1)(x+1) \\
        &= (x+1)(x^2 + 2x + 1) \\
        &= x^3 + 3x^2 + 3x + 1
\end{align*}
with the \bftt{align*} environment.

% for whatever reason, this doesn't play well with the twoup environment
\begin{minted}[fontsize=\small,frame=single]{latex}
\begin{align*}
(x+1)^3 &= (x+1)(x+1)(x+1) \\
        &= (x+1)(x^2 + 2x + 1) \\
        &= x^3 + 3x^2 + 3x + 1
\end{align*}
\end{minted}
\item An ampersand \keystrokebftt{\&} separates the left column (before the
$=$) from the right column (after the $=$).
\item A double backslash \keystrokebftt{\bs}\keystrokebftt{\bs} starts a new
line.
}\end{itemize}
\end{frame}


%%%%%%%%%%%%%%%%%%%%%%%%%%%%%%%%%%%%%%%%%%%%%%%%%%%%%%%%%%%%%%%%%%%%%%%%%%%%%%%
%%%%%%%%%%%%%%%%%%%%%%%%%%%%%%%%%%%%%%%%%%%%%%%%%%%%%%%%%%%%%%%%%%%%%%%%%%%%%%%
%%%%%%%%%%%%%%%%%%%%%%%%%%%%%%%%%%%%%%%%%%%%%%%%%%%%%%%%%%%%%%%%%%%%%%%%%%%%%%%
\begin{frame}[fragile]{Typesetting Exercise 2}

\begin{block}{Typeset this in \LaTeX:}
Let $X_1, X_2, \ldots, X_n$ be a sequence of independent and identically
distributed random variables with $\operatorname{E}[X_i] = \mu$ and
$\operatorname{Var}[X_i] = \sigma^2 < \infty$, and let
\begin{equation*}
S_n = \frac{1}{n}\sum_{i=1}^{n} X_i
\end{equation*}
denote their mean. Then as $n$ approaches infinity, the random variables
$\sqrt{n}(S_n - \mu)$ converge in distribution to a normal $N(0, \sigma^2)$.
\end{block}
\vskip 2ex
\begin{center}
\fbox{\href{\wlnewdoc{basics-exercise-2.tex}}{%
Click to open this exercise in \wllogo{}}}
\end{center}
\begin{itemize}
\item Hint: the command for $\infty$ is \cmdbs{infty}.
\item Once you've tried,
\fbox{\href{\wlnewdoc{basics-exercise-2-solution.tex}}{%
click here to see my solution}}.
\end{itemize}
\end{frame}

%%%%%%%%%%%%%%%%%%%%%%%%%%%%%%%%%%%%%%%%%%%%%%%%%%%%%%%%%%%%%%%%%%%%%%%%%%%%%%%
%%%%%%%%%%%%%%%%%%%%%%%%%%%%%%%%%%%%%%%%%%%%%%%%%%%%%%%%%%%%%%%%%%%%%%%%%%%%%%%
%%%%%%%%%%%%%%%%%%%%%%%%%%%%%%%%%%%%%%%%%%%%%%%%%%%%%%%%%%%%%%%%%%%%%%%%%%%%%%%
\begin{frame}{End of Part 1}
\begin{itemize}
\item Congrats! You've already learned how to \ldots
\begin{itemize}
\item Typeset text in \LaTeX.
\item Use lots of different commands.
\item Handle errors when they arise.
\item Typeset some beautiful mathematics.
\item Use several different environments.
\item Load packages.
\end{itemize}
\item That's amazing!
\item In Part 2, we'll see how to use \LaTeX{} to write structured documents
with sections, cross references, figures, tables and bibliographies. See you
then!
\end{itemize}

\end{frame}

\end{document}
