\documentclass{article}
\usepackage{amsmath}
\begin{document}

Sia $X_1, X_2, \ldots, X_n$ una sequenza di variabili aleatorie indipendenti e distribuite identicamente con media $\operatorname{E}[X_i] = \mu$ e varianza
$\operatorname{Var}[X_i] = \sigma^2 < \infty$, e sia
\begin{equation*}
S_n = \frac{1}{n}\sum_{i=1}^{n} X_i
\end{equation*}
la media. Quando $n$ tende all'infinito, le variabili aleatorie
$\sqrt{n}(S_n - \mu)$ convergono verso una distribuzione normale $N(\mu, \sigma^2)$.

% Per rappresentare la N della distribuzione normale si puo' 
% usare un carattere calligrafico particolare con il
% seguente comando $\mathcal{N}(\mu, \sigma^2)$.

\end{document}
