\documentclass[12pt]{article}

\usepackage{url}

\begin{document}

Dieci segreti per fare una bella presentazione scientifica
Autore: Il lettore


Introduzione

Il testo di questo esercizio è una versione notevolmente ridotta e leggermente modificata dell'ottimo articolo omonimo (in inglese) di Mark Schoeberl e Brian Toon:
\url{http://www.cgd.ucar.edu/cms/agu/scientific_talk.html} e poi tradotta e modificata dal docente del corso.

I Segreti

Ho compilato questo elenco personale di "Segreti" ascoltando presentazioni efficaci e terribili. Non pretendo che questo elenco sia completo: sono sicuro che ci sono cose che ho tralasciato. Ma la mia lista probabilmente copre circa il 90% del da farsi.

1) Prepara il tuo materiale in modo accurato e logico. Strutturalo come una storia. Non ci sono scuse per la mancanza di preparazione.

2) Esercitati e ripeti la tua presentazione, magari davanti a uno specchio. 

3) Non inserire troppo materiale. Pochi punti ben detti passano molte più informazioni di un minestrone.

4) Evita le formule ogni qual volta puoi. Si dice che per ogni formula nella tua presentazione, il numero di persone ti seguono si dimezza. Quindi, se q è il numero di formule nella presentazione, n è il numero di persone presenti e m quelle che capiscono, abbiamo che:

m = (n diviso 2 alla q)


5) La conclusione deve essere di pochi punti: non si riescono a ricordare più di un paio di concetti da una presentazione.

6) Parla al pubblico e non allo schermo.

7) Evita di produrre suoni fastidiosi. Cerca di evitare "Ummm" o "Ahhh" tra le frasi.

8) Ripulisci per bene la tua struttura grafica. Ecco un elenco di suggerimenti per una grafica migliore:

* Usa lettere grandi.

* Mantieni la grafica semplice. Non mostrare ciò che non ti serve.

* Usa il colore.

9) Sii affabile ed empatico nel rispondere alle domande.

10) Usa l'umorismo ogni qual volta possibile. 



\end{document}
